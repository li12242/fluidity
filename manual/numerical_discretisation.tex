\chapter{Numerical discretisation}\label{chap:numerical_discretisation}
\section{Introduction \& some definitions}\label{Sect:ND_Intro}

In this chapter we define the domain we are solving over as $\Omega$. The boundary to
$\Omega$ is defined as $\dOmega$
%(NB. $\Gamma$ is another common symbol for a boundary) 
and can be split into different sections, e.g.
$\partial\Omega = \dOmega^{N}\cup \dOmega^{D}$
where $\dOmega^{N}$ is that part of the boundary over
which Neumann conditions are applied and $\dOmega^{D}$ is that part of the boundary over
which Dirichlet conditions are applied. For clarity a subscript showing the field in question will
be given on the $N$ and $D$ characters.

The unit vector \vec{n} is always assumed to be the
outward facing normal vector to the domain.
In the following this notation is used to describe both the normal vector at the boundary
and between elements in the interior.

\section{Spatial discretisation of the advection-diffusion equation}
\label{Sect:ND_advection_diffusion_discretisation}

The advection-diffusion equation for a scalar tracer $c$,
is given in conservative form by
\begin{equation}\label{eq:advdif}
  \ppt{c} + \div\left(\vec{u} c\right) - \div\tensor{\mu}\cdot\grad c = 0
\end{equation}.

We now define a partition of the boundary such that
$\partial\Omega = \dOmega^{N}\cup \dOmega^{D}$, and impose boundary conditions on $c$:
\begin{gather}
  \label{eq:cboundary}
  \vec{n}\cdot\tensor{\mu}\cdot\grad c=g_{N_c} \quad\textrm{on}\quad \dOmega^{N_c},\\
  c=g_{D_c} \quad\textrm{on}\quad \dOmega^{D_c}.
\end{gather}

\subsection{Continuous Galerkin discretisation}
\label{sect:ND_advection_diffusion_CG}

In the continuous Galerkin (sometimes we may use the shorthand CG, which should not be confused with conjugate gradients) method,
the solution fields are constrained to be
continuous between elements. The fields are only assumed to be $C^0$
continuous, that is to say there is no assumption that the gradient of a
field is continuous over element boundaries.

\subsubsection{Weak form}
\index{weak form}
\index{test function}
\index{advection-diffusion equation!weak form}

Development of the finite element method begins by writing the equations in
\emph{weak form}.  The weak form of the advection-diffusion equation is
obtained by pre-multiplying it with a test function $\phi$
and integrating over the domain $\Omega$, such that
\begin{equation}
  \int_\Omega \phi \left( \ppt{c} + \vec{u}\cdot\grad c -
    \div\tensor{\mu}\cdot \grad c\right) = 0.
\end{equation}
Integrating the advection and diffusion terms by parts yields
\begin{equation}\label{eq:weak_adv_diff}
  \int_\Omega
  \phi \ppt{c}
  -\grad\phi\cdot\vec{u} c +
  \grad\phi\cdot \tensor{\mu}\cdot\grad c
  +\int_\dOmega \vec{n}\cdot \vec{u} c
  - \vec{n}\cdot\tensor{\mu}\cdot\grad c
  = 0.
\end{equation}
For simplicity sake let us first assume the boundaries are closed
($\vec{u}\cdot\vec{n}=0$) and we apply a homogeneous Neumann boundary
condition everywhere, such that $\partial c/\partial n=0$, which gives
\begin{equation}\label{eq:weak_adv_diff_simp}
  \int_\Omega \phi \frac{\partial c}{\partial t}
    -\grad\phi\cdot\vec{u} c +
    \grad\phi\cdot \tensor{\mu}\cdot\grad c = 0.
\end{equation}
Note that due to the Neumann boundary condition the boundary term has
dropped out. The tracer field $c$ is now called a \emph{weak solution} of
the equations if \eqref{eq:weak_adv_diff_simp} holds true for all $\phi$ in some
space of test functions $V$. The choice of a suitable test space $V$ is dependent
on the equation (for more details see \citet{elman2005}).

\index{Sobolev space} An important observation is that
\eqref{eq:weak_adv_diff} only contains first derivatives of the field $c$,
so that we can now include solutions that do not have a continuous second
derivative. For these solutions the original equation \eqref{heat} would not
be well-defined. These solutions are termed \emph{weak} as they do not
have sufficient smoothness to be \emph{classical} solutions to the problem.
All that is required for the weak solution is that the first derivatives of $c$
can be integrated along with the test function. A more precise definition of
this space, the \emph{Sobolev space}, can again be found \citet{elman2005}.

\subsubsection{Finite element discretisation}
\index{trial function}
Instead of looking for a solution in the entire function (Sobolev) space,
in finite element methods discretisation is performed by restricting the solution to a
finite-dimensional subspace. Thus the solution can be
written as a linear combination of a finite number of functions,
the \emph{trial functions} $\phi_i$ that form a basis of the
\emph{trial space}, defined such that
\begin{equation*}
  c(\vec{x})=\sum_i c_i \phi_i(\vec(x)).
\end{equation*}
The coefficients $c_i$ can be written in vector format. The
dimension of this vector equals the dimension of the trial
space. In the sequel any function in
this way represented as a vector will be denoted as $\dvec{c}$.

\index{Galerkin!methods}
\index{Galerkin!projection}
Since the set of trial functions is now much smaller (or rather finite as opposed to infinite-dimensional), we also need
a much reduced set of test functions for the equation in weak
form \eqref{eq:weak_adv_diff_simp} in order to find a unique
solution. A common choice is, in fact, to choose the same
test and trial space. Finite element methods that make this choice are
referred to as \emph{Galerkin methods} --- the discretisation
can be seen as a so called \emph{Galerkin
projection} of the weak equation to a finite subspace.

\index{PN@\PN}
\index{Galerkin!continuous}
\index{Galerkin!discontinuous}
There are many possibilities for choosing the finite-dimensional trial and test spaces.
A straightforward choice is to restrict the functions to be polynomials of degree
$n\leq N$ within each element. These are referred to as \PN discretisations.
As we generally need functions for which the first
derivatives are integrable, a further restriction
is needed. If we allow the functions to be any polynomial of
degree $n\leq N$ within the elements the function can be
discontinuous in the boundary between elements.
Continuous Galerkin methods therefore
restrict the test and trial functions to arbitrary polynomials
that are continuous between the elements. Discontinuous Galerkin methods,
that allow any polynomial, are also possible but require
extra care when integrating by parts (see section \ref{sec:NM_DG_advection}).

\index{P1@\Pone}
\index{basis function}
If we choose \Pone as our test and trial functions, \ie piecewise linear
functions, within each element we only need to know the value
of the function at 3 points in 2D, and 4 points in 3D.
In \fluidity\ these points are chosen to be the vertices of
the triangles (in 2D) or tetrahedra (in 3D) tessellating the domain.
For continuous Galerkin the continuity
requirement then comes down to requiring
the functions to have a single value at each
vertex. A set of basis functions $\phi_i$
for this space is easily found by choosing the piecewise linear functions
$\phi_i$ that satisfy:
\begin{gather*}
  \phi_i(x_i)=1, \;\forall i\\
  \phi_i(x_{j\neq i})=0,\;\forall i,j,
\end{gather*}
where $x_i$ are the vertices in the mesh.
This choice of basis functions has the following useful property:
\begin{equation*}
  c_i=c(\vec{x}_i),\quad \text{for all nodes $x_i$ in the mesh.}
\end{equation*}
This naturally describes trial functions that are linearly
interpolated between the values $c_i$ in the nodes.
Higher order polynomials can be represented using more
nodes in the element (see Figure \ref{fig:cgshapefunctions}).

\begin{figure}[btp]
\begin{center}
\begin{tabular}{lr}
\xfig{numerical_discretisation_images/P1cgshapefunction1d} & \xfig{numerical_discretisation_images/P2cgshapefunction1d} \\
\xfig{numerical_discretisation_images/P1cgshapefunction2d} & \xfig{numerical_discretisation_images/P2cgshapefunction2d}
\end{tabular}
\caption{One-dimensional (a, b) and two-dimensional (c, d) schematics of piecewise linear (a, c) and piecewise quadratic (b, d) continuous shape functions.  The shape function has value $1$ at node $A$ descending to $0$ at all surrounding nodes.  The number of nodes per element, $e$, depends on the polynomial order while the support, $s$, extends to all the elements surrounding node $A$.}
\label{fig:cgshapefunctions}
\end{center}
\end{figure}

As discussed previously the test space in Galerkin finite element methods is the same as the
trial space. So for \PN the test functions can be an arbitrary linear combination
of the same set of basis functions. To make sure that the equation we are solving
integrates to zero with all such test functions, all we have to do is make sure
that the equation tested with the basis functions integrate to zero. The discretised
version of \eqref{eq:weak_adv_diff_simp} therefore becomes
\index{advection-diffusion equation!weak form!discretised}
\begin{equation}
  \sum_j \left\{ \int_\Omega \phi_i \phi_j  \ddt{c_j} -
    \grad\phi_i\cdot\vec{u}\phi_j  c_j +
    \grad\phi_i\cdot \tensor{\mu}\cdot\grad \phi_j c_j \right\} = 0,
    \quad\text{for all }\phi_i.
\end{equation}
where we have substituted $c=\sum_j \phi_j c_j$. From this it is readily seen that
we have in fact obtained a matrix equation of the following form
\begin{equation}
  \mat{M} \ddt{\dvec{c}}+\mat{A}(\vec{u})\dvec{c}+\mat{K}\dvec{c}=0,
  \label{eq:cg_adv_diff_mat}
\end{equation}
where $\mat{M}, \mat{A}$ and $\mat{K}$ are the following matrices
\begin{equation}
  \mat{M}_{ij}=\int_\Omega \phi_i\phi_j, \quad
  \mat{A}_{ij}=-\int_\Omega \grad\phi_i\cdot\vec{u} \phi_j, \quad
  \mat{K}_{ij}=\int_\Omega \grad\phi_i\cdot \tensor{\mu}\cdot\grad \phi_j.
\end{equation}

\subsubsection{Advective stabilisation for CG}
\label{Sect:ND_advective_stabilisation_CG}
\index{advection-diffusion equation!continuous Galerkin}
\index{Galerkin!continuous!advection}
\index{stabilisation!advection}

It is well known that a continuous Galerkin discretisation of an
advection-diffusion equation for an advection dominated flow can suffer from
over- and under-shoot errors. Furthermore, these overshoot errors are not
localised: they can propagate throughout the simulation domain and pollute the
global solution \citep{hughes1987}. Consider a simple 1D linear steady-state
advection-diffusion problem for a single scalar $c$:

\begin{equation}\label{eqn:trivial_advdif}
  u \ppx{c} - \kappa \ppxx{c} = f(x),
\end{equation}
or equivalently in weak form:
\begin{equation}\label{eqn:trivial_advdif_weak}
  \int_\Omega \phi (u \ppx{c} + \kappa \ppx{\phi} \ppx{c} - f(x)) = 0,
\end{equation}
where we have integrated by parts and applied the natural Neumann boundary
condition $\partial c / \partial x = 0$ on $\partial \Omega$.
Discretising \eqref{eqn:trivial_advdif_weak} with a continuous Galerkin method
leads to truncation errors in the advection term equivalent to a negative
diffusivity term of magnitude \citep{DoneaBook}:
\begin{equation}\label{eqn:cg_implicit_diffusivity}
  \bar{\kappa} = \xi \kappa \Pe,
\end{equation}
where:
\begin{equation}\label{eqn:xi_parameter}
  \xi = \frac{1}{\tanh(\Pe)} - \frac{1}{\Pe},
\end{equation}
and:
\begin{equation}\label{eqn:grid_pe}
  \Pe = \frac{u h}{2 \kappa},
\end{equation}
is a grid P\'eclet number, with grid spacing $h$.
\index{P\'eclet number!grid}

This implicit negative diffusivity becomes
equal to the explicit diffusivity at a P\'eclet greater than one, and hence
instability occurs for $\Pe \geq 1$. In order to achieve a stable discretisation
using a continuous Galerkin method one is therefore required either to increase
the model resolution so as to reduce the grid P\'eclet number, or to apply
advective stabilisation methods.

\paragraph{Balancing diffusion}\label{sec:balancing_diffusion}
\index{stabilisation!advection!balancing diffusion}

A simple way to stabilise the system is to add an extra diffusivity of
equal magnitude to that introduced by the discretisation of the advection term,
but of opposite sign. This method is referred to as \textit{balancing diffusion}.
Note, however, that for two or more dimensions, we require this balancing
diffusion to apply in the along-stream direction only \citep{brooks1982, DoneaBook}.
For this reason this method is also referred to as \textit{streamline-upwind}
stabilisation. The multidimensional weak-form equation:
\begin{equation}\label{eqn:trivial_advdif_weak_md}
  \int_\Omega \phi (\vec{u} \cdot \grad c + \kappa \grad \phi \grad c -f(x)) = 0,
\end{equation}
is therefore modified to include an additional artificial balancing diffusion
term \citep{DoneaBook}:
\begin{equation}\label{eqn:balancing_diffusion}
  \int_\Omega \phi (\vec{u} \cdot \grad c + \kappa \grad \phi \grad c - f(x)) +
  \int_\Omega \frac{\bar{\kappa}}{\left|\left| \vec{u} \right|\right|^2}
  (\vec{u} \cdot \grad \phi) (\vec{u} \cdot \grad \vec{c})
  = 0.
\end{equation}

The exact form of the multidimensional stability parameter $\bar{\kappa}$
is a research issue. See \ref{sec:stabilisation_parameter} for implementations in
\fluidity.

The addition of the balancing diffusion term combats the negative implicit
diffusivity of the continuous Galerkin method. However, we are no longer solving
the original equation - for a pure advection flow we are now solving a modified
version of the original advection-diffusion equation with the grid
P\'eclet number artificially reduced from
infinity to unity everywhere. Hence streamline-upwind is not a \textit{consistent}
stabilisation method, and there is a reduction in the degree of numerical
convergence.

\paragraph{Streamline-upwind Petrov-Galerkin}\label{sec:supg}
\index{Galerkin!Petrov-}
\index{Petrov-Galerkin}
\index{stabilisation!advection!Petrov-Galerkin}

The streamline-upwind stabilisation method can be extended to a consistent (and
hence high order accurate) method by introducing stabilisation in the form of a weighted
residual \citep{DoneaBook}:
\begin{equation*}
  \int_\Omega \phi (\vec{u} \cdot \grad c + \kappa \grad \phi \grad c - f(x)) +
  \int_\Omega \tau P(\phi) R(\phi)
  = 0,
\end{equation*}
where:
\begin{equation*}
  R(\phi) = \vec{u} \cdot \grad c - \kappa \grad^2 c - f(x),
\end{equation*}
is the equation residual, $\tau$ is a stabilisation parameter and $P(\phi)$ is
some operator.
Note that this is equivalent to replacing the test function in the original
equation with $\tilde{\phi} = \phi + \tau P(\phi)$.
Looking at equation \eqref{eqn:balancing_diffusion}, it can seen that the
balancing diffusion term is equivalent to replacing the test function for the
advection term with:
\begin{equation}\label{eqn:supg_test_function}
  \tilde{\phi} = \phi + \frac{\bar{\kappa}}{\left|\left| \vec{u} \right|\right|^2} u \cdot \grad.
\end{equation}

This suggests a stabilisation method whereby the test function in the advection-diffusion
equation is replaced with the test function in \eqref{eqn:supg_test_function}.
This approach defines the \textit{streamline-upwind Petrov-Galerkin} (SUPG) method. The
weighted residual formulation of this method guarantees consistency, and hence
preserves the accuracy of the method. Furthermore, while this method does
still possess under- and over-shoot errors in the presence of sharp solution
gradients, these errors remain localised \citep{hughes1987}.

\paragraph{Stabilisation parameter}\label{sec:stabilisation_parameter}

Note that, as mentioned in \ref{sec:balancing_diffusion}, the choice of
stabilisation parameter $\bar{\kappa}$ is somewhat arbitrary. \fluidity\ implements \citep{brooks1982, DoneaBook}:

\begin{equation}\label{eqn:md_nu_bar}
  \bar{\kappa} = \frac{1}{2} \sum{\xi_i u_i h_i},
\end{equation}

where the summation is over the quadrature points of an individual element. The
grid spacings $h_i$ are approximated from the elemental Jacobian.

As an alternative, \citet{raymond1976} show that for
1D transient pure-advection a choice of:

\begin{equation}\label{eqn:md_nu_bar_transient}
  \bar{\kappa} = \frac{1}{\sqrt{15}} \sum{\xi_i u_i h_i},
\end{equation}

minimises phase errors.

Computing the $\xi$ factor at quadrature points is potentially expensive due to
the evaluation of a hyperbolic-tangent \eqref{eqn:xi_parameter}. Sub-optimal but more computationally
efficient approximations for $\xi$ are the \emph{critical rule} approximation \citep{brooks1982}:

\begin{equation}
  \xi = \begin{cases}
          -1 - 1 / \Pe  & \Pe < -1 \\
          0             & -1 \le \Pe \le 1 \\
          1 + 1 / \Pe   & \Pe > 1,
        \end{cases}
\end{equation}

and the \emph{doubly-asymptotic} approximation \citep{DoneaBook}:

\begin{equation}
  \xi = \begin{cases}
          \Pe / 3  & \left| \Pe \right| \le 3 \\
          \sgn(\Pe)  & \textrm{otherwise}.
        \end{cases}
\end{equation}

\paragraph{Implementation limitations}

The SUPG implementation in \fluidity\ does not modify the test function derivatives
or the face test functions. Hence the SUPG implementation is only consistent
for degree one elements with no non-zero Neumann or weak Dirichlet boundary
conditions.

SUPG is considered ready for production use for scalar advection-diffusion
equation discretisation, but is still experimental for momentum discretisation.

\subsubsection{Example}

The following example considers pure advection of a 1D top hat of unit magnitude
and width 0.25 in a periodic domain of unit size. The top hat is advected with a
Courant number of $1 / 8$. Figure \ref{fig:top_hat_cg} shows the solution after
80 timesteps using a continuous Galerkin discretisation. Figure \ref{fig:top_hat_su}
shows the solution when streamline-upwind stabilisation is applied.
Figure \ref{fig:top_hat_supg} shows the solution when streamline-upwind
Petrov-Galerkin is applied, using a stabilisation parameter as in \eqref{eqn:md_nu_bar}.

\begin{figure}[ht]
  \centering
  \fig[width=0.5\textwidth]{numerical_discretisation_images/advective_stabilisation/cg}
  \caption{Pure advection of a 1D top hat at CFL number $1 / 8$ after 80 timesteps
           using a continuous Galerkin discretisation.}
  \label{fig:top_hat_cg}
\end{figure}

\begin{figure}[ht]
  \centering
  \fig[width=0.5\textwidth]{numerical_discretisation_images/advective_stabilisation/su}
  \caption{Pure advection of a 1D top hat at CFL number $1 / 8$ after 80 timesteps
           using a continuous Galerkin discretisation with streamline-upwind
           stabilisation.}
  \label{fig:top_hat_su}
\end{figure}

\begin{figure}[ht]
  \centering
  \fig[width=0.5\textwidth]{numerical_discretisation_images/advective_stabilisation/supg}
  \caption{Pure advection of a 1D top hat at CFL number $1 / 8$ after 80 timesteps
           using a continuous Galerkin discretisation with streamline-upwind
           Petrov-Galerkin stabilisation.}
  \label{fig:top_hat_supg}
\end{figure}

\subsection{Boundary conditions}
\index{boundary conditions!Neumann!weakly imposed}
In the derivation of \eqref{eq:weak_adv_diff} we have assumed a homogeneous Neumann
boundary condition on all boundaries. If we are considering all possible
solutions $c$, the boundary term we have left out is
\begin{equation}\label{eq:missing_boundary_term}
  \int_{\partial\Omega} \phi~\vec{n}\cdot\tensor{\mu}\cdot\grad c.
\end{equation}
A natural way of imposing an inhomogeneous boundary Neumann boundary condition
\begin{equation*}
  \vec{n}\cdot\tensor{\mu}\cdot\grad c=g_N,
\end{equation*}
where $g_N$ can be any function on the boundary $\partial\Omega$, is to impose it
weakly. This is done in the same way as before:
\begin{equation}\label{eq:weak_neumann}
  \int_{\partial\Omega} \phi~\vec{n}\cdot\tensor{\mu}\cdot\grad c=
    \int_{\partial\Omega} \phi~g_N, \quad \text{for all }\phi.
\end{equation}
Thus \eqref{eq:weak_neumann} can be used to replace the missing
boundary term \eqref{eq:missing_boundary_term} with an integral of $\phi~g_N$
over the boundary.

\index{boundary conditions!Dirichlet!weakly imposed}
In a similar way, a weakly imposed Dirichlet boundary condition can be related to an
integration by parts of the advection term. Let us consider a pure advection problem
($\mu=0$). The weak form of this equation integrated by parts reads:
\begin{equation*}
  \int_\Omega \phi \frac{\partial c}{\partial t} -
    \left(\grad\cdot \phi~\vec{u}\right)~c +
    \int_{\partial\Omega} \vec{n}\cdot\vec{u}~c
    = 0.
\end{equation*}
The last (boundary) term can again be substituted by a weakly imposed
boundary condition $c=g_D$.  In this case however we only want to impose
this on the inflow boundary, and the original term remains for the outflow
boundary:
\begin{equation}\label{eq:adv_integrated_by_parts}
  \int_\Omega \phi \frac{\partial c}{\partial t} -
    \left(\grad\cdot \phi~\vec{u}\right)~c +
    \int_{\partial\Omega_-} \vec{n}\cdot\vec{u}~g_D +
    \int_{\partial\Omega_+} \vec{n}\cdot\vec{u}~c
    = 0,
\end{equation}
where $\partial\Omega_-$ and $\partial\Omega_+$ refer to respectively
the inflow ($\vec{n}\cdot\vec{u}>0$) and outflow boundaries ($\vec{n}\cdot\vec{u}<0$).

It is to be noted that when we are applying boundary conditions weakly we
still consider the full set of test functions, even those that don't satisfy
the boundary condition. This means the discrete solution will not satisfy
the boundary condition exactly. Instead the solution will converge to the
right boundary condition along with the solution in the interior as the mesh
is refined.

\index{boundary conditions!Dirichlet!strongly imposed} An alternative way of
implementing boundary conditions, so called \emph{strongly imposed} boundary
conditions, is to restrict the trial space to only those functions that
satisfy the boundary condition. In the discrete trial space this means we no
longer allow the coefficients $c_i$ that are associated with the nodes $x_i$
on the boundary, to vary but instead substitute the imposed Dirichlet
boundary condition. As this decreases the dimension of the trial space, we
also need to limit the size of the test space. This is simply done by
removing the test function $\phi_i$ associated with the nodes $x_i$ on the
boundary, from the test space. Although this guarantees that the Dirichlet
boundary condition will be satisfied exactly, it does not at all mean that
the discrete solution converges to the exact continuous solution more
quickly than it would with weakly imposed boundary conditions. Strongly
imposed boundary conditions may sometimes be necessary if the boundary
condition needs to be imposed strictly for physical reasons.

\subsection{Discontinuous Galerkin discretisation}\label{sec:NM_DG_advection}
\index{Galerkin!discontinuous!advection}
\index{advection-diffusion equation!discontinuous Galerkin}

Integration by parts can be used to avoid taking
derivatives of discontinuous functions.
When using discontinuous test \emph{and} trial functions (see Figure \ref{fig:dgshapefunctions}) however,
neither the original advection equation, \eqref{eq:weak_adv_diff} with $\mu=0$,
nor the version \eqref{eq:adv_integrated_by_parts} integrated by parts are well-defined.
Within an element $e$ however the functions are continuous, and everything is well defined. So within
a single element we may write
\begin{equation}\label{eq:adv_diff_dg}
  \int_e \phi \frac{\partial c}{\partial t} -
    \left(\grad\cdot \phi~\vec{u}\right)~c +
    \grad\phi\cdot\tensor{\mu}\cdot\grad c +
    \int_{\partial e} \phi\widehat{\vec{n}\cdot\vec{u}~c} -
    \phi\widehat{\vec{n}\cdot\tensor{\mu}\cdot\grad c}
    = 0,
\end{equation}
The hatted terms represent fluxes across the element facets: and therefore
from one element to the other. Due to the discontinuous nature of the
fields, there is no unique value for these flux terms, however the
requirement that $c$ be a conserved quantity does demand that adjacent
elements make a consistent choice for the flux between them. The choice of flux
schemes therefore forms a critical component of the discontinuous Galerkin
method.

\begin{figure}[tbp]
\begin{center}
\begin{tabular}{lr}
\xfig{numerical_discretisation_images/P1dgshapefunction1d} & \xfig{numerical_discretisation_images/P2dgshapefunction1d} \\
\xfig{numerical_discretisation_images/P1dgshapefunction2d} & \xfig{numerical_discretisation_images/P2dgshapefunction2d}
\end{tabular}
\caption{One-dimensional (a, b) and two-dimensional (c, d) schematics of piecewise linear (a, c) and piecewise quadratic (b, d) discontinuous shape functions.  The shape function has value $1$ at node $A$ descending to $0$ at all surrounding nodes.  The number of nodes per element, $e$, depends on the polynomial order while the support, $s$, covers the same area as the element, $e$.}
\label{fig:dgshapefunctions}
\end{center}
\end{figure}

The application of boundary conditions occurs in the same manner as for the
continuous Galerkin method. The complete system of equations is formed
by summing over all the elements. Assuming weakly applied boundary
conditions, this results in:
\begin{multline}
  \sum_e \left\{ \int_e \phi \frac{\partial c}{\partial t}
  - \left(\grad\cdot \phi~\vec{u}\right)~c
  + \grad\phi\cdot\tensor{\mu}\cdot\grad c \right. \\
  \left. + \int_{\partial e\,\cap\dOmega^{D_c}_{-}} \phi\vec{n}\cdot\vec{u}~g_D
  + \int_{\partial e\,\cap\dOmega^{D_c}_{+}} \phi\vec{n}\cdot\vec{u}~c
  - \int_{\partial e\,\cap\dOmega^{D_c}}
  \phi\vec{n}\cdot\tensor{\mu}\cdot\grad c \right. \\
  \left. + \int_{\partial e\,\cap\dOmega^{N_c}} \phi\vec{n}\cdot\vec{u}~c
  - \phi\vec{n}\cdot\tensor{\mu}\cdot\grad g_N + \int_{\partial e\,\setminus\dOmega} \phi\widehat{\vec{n}\cdot\vec{u}~c}
  - \phi\widehat{\vec{n}\cdot\tensor{\mu}\cdot\grad c} \right\}
    = 0.
\end{multline}

\subsubsection{Discontinuous Galerkin advection}
\label{Sect:ND_discontinuous_galerkin_advection}

Consider first the case in which $\tensor{\mu}=0$. In this case, equation
\eqref{eq:adv_diff_dg}\ reduces to:
\begin{equation}\label{eq:adv_dg}
  \int_e \phi \frac{\partial c}{\partial t} -
    \left(\grad\cdot \phi~\vec{u}\right)~c +
    \int_{\partial e} \phi\widehat{\vec{n}\cdot\vec{u}~c}
    = 0,
\end{equation}
and the question becomes, how do we represent the flux
$\widehat{\vec{n}\cdot\vec{u}~c}$\ ?

\fluidity\ supports two different advective fluxes for DG. Upwind and local
Lax-Friedrichs. For each flux scheme, there are two potentially
discontinuous fields for which a unique value must be chosen. The first is
the advecting velocity $\vec{u}$. The default behaviour is to average the
velocity on each side of the face. The velocity is averaged at each
quadrature point so decisions on schemes such as upwinding are made on a
per-quadrature point basis. The second scheme is to apply a Galerkin
projection to project the velocity onto a continuous basis. This amounts to
solving the following equation:
\begin{equation}
  \int_\Omega \vec{\hat{\psi}} \cdot\vec{\hat{u}}
  = \int_\Omega \vec{\hat{\psi}}\cdot \vec{u}
\end{equation}
where the hatted symbols indicate that the quantity in question is
continuous between elements. In the following sections, $\vec{\hat{u}}$ will
be used to indicate the flux velocity, which will have been calculated with
one of these methods. Note that using the averaging method,
$\vec{\hat{u}}=\vec{u}$ on the interior of each element with only the inter-element
flux differing from $\vec{u}$ while for the projection method, $\vec{\hat
  u}$ and $\vec{u}$ may differ everywhere.

\paragraph{Upwind Flux}

In this case, the value of $c$ at each quadrature point on the face is taken
to be the upwind value. For this purpose the upwind value is as follows: if
the flow is out of the element then it is the value on the interior side of
the face while if the flow is into the element, it is the value on the
exterior side of the face. If we denote the value of $c$ on the upwind side
of the face by $c_{\mathrm{upw}}$ then equation \eqref{eq:adv_dg}\ becomes:
\begin{equation}
  \int_e \phi \ppt{c} -
    \left(\grad\cdot \phi~\vec{\hat u}\right)~c +
    \int_{\partial e} \phi\vec{n}\cdot\vec{\hat u}~c_{\mathrm{upw}}
    = 0,
\end{equation}
Summing over all elements, including boundary conditions and writing
$c_{\mathrm{int}}$ to indicate flux terms which use the value of $c$ from the
current element only, we have:
\begin{equation}
  \begin{split}
    \sum_e \left\{ \int_e \phi \ppt{c}
    - \left(\grad\cdot \phi~\vec{\hat u}\right)~c \right.
    &+ \left. \int_{\partial e \cap\partial\Omega^{D_c}_-} \vec{n}\cdot\vec{\hat u}~g_D \right. \\
    &+ \left. \int_{\partial e \cap\partial\Omega^{N_c}\cap\partial\Omega^{D_c}_+} \vec{n}\cdot\vec{\hat u}~c_{\mathrm{int}} \right. \\
    &+ \left. \int_{\partial e\setminus (\partial\Omega_D\cap\partial\Omega_N)}
    \vec{n}\cdot\vec{\hat u}~c_{\mathrm{upw}} \right\} =0.
    \label{eq:dg_adv_diff_integrated_once}
  \end{split}
\end{equation}

\subparagraph{Second integration by parts}

In \fluidity\ the advection term with upwinded flux may be subsequently
integrated by parts again within each element. As this is just a local
operation on the continuous pieces of $c$ within each element, the new
boundary integrals take the value of $c$ on the inside of the element,
$c_{\mathrm{int}}$. On the outflow boundary of each element this means the
$c_{\mathrm{int}}$ cancels against $c_{\mathrm{upw}}$ (when the summation over all elements occurs).
Writing $\partial e_-$ for the inflow part of the element boundary, we therefore obtain:
\begin{equation}
\begin{split}
  \sum_e \left\{ \int_e \phi \ppt{c}
  - \phi~\vec{\hat u}\cdot\grad c \right.
    +& \left. \int_{\partial e_- \cap\partial\Omega^{D_c}_-} \vec{n}\cdot\vec{\hat u}~(g_D -c_{\mathrm{int}}) \right. \\
    +& \left. \int_{\partial e_-\setminus (\partial\Omega_D\cap\partial\Omega_N)}
    \vec{n}\cdot\vec{\hat u}~
      (c_{\mathrm{upw}}-c_{\mathrm{int}}) \right\} =0.
    \label{eq:dg_adv_diff_integrated_twice}
\end{split}
\end{equation}
The difference $c_{\mathrm{int}}-c_{\mathrm{upw}}$ on the inflow boundary
remains, and is often referred to as a \emph{jump condition}. Note also that
the boundary terms on the Neumann domain boundary and the outflow part of
the Dirichlet boundary (really also a Neumann boundary) have disappeared.
Note that \eqref{eq:dg_adv_diff_integrated_once} and
\eqref{eq:dg_adv_diff_integrated_twice} are completely equivalent. The
advantage of the second form, referred to in \fluidity\ as
``integrated-by-parts-twice'', is that the numerical evaluation of the
integrals (quadrature), may be chosen not to be exact (incomplete quadrature). For this
reason the second form may be more accurate as the internal outflow boundary
integrals are cancelled exactly.

\paragraph{Local Lax-Friedrichs flux}

The local Lax-Friedrichs flux formulation is defined in
\citet[p208]{cockburn2001}. For the
particular case of tracer advection, this is given by:
\begin{equation}
  \widehat{\vec{n}\cdot\vec{u}~c}=\frac{1}{2}\vec{n}\cdot\vec{\hat u}
  \left(c_{\mathrm{int}}+c_{\mathrm{ext}}\right)
  -\frac{C}{2}c_{\mathrm{int}}-c_{\mathrm{ext}}
\end{equation}
where $c_{\mathrm{ext}}$ is the value of $c$ on the exterior side of the
  element boundary and in which for each facet $s\subset\partial e$:
\begin{equation}
  C= \sup_{x\in s}|\vec{\hat u}\cdot\vec{n}|
\end{equation}

\subsubsection{Advective stabilisation for DG}
\index{stabilisation!advection!discontinuous Galerkin}
\index{Galerkin!discontinuous!slope limiters}
\index{discontinuous Galerkin!see{Galerkin!discontinuous}}
\index{continuous Galerkin!see{Galerkin!continuous}}

As described by \cite{cockburn2001}, the DG method with $p$-th order
polynomials using an appropriate Riemann flux (the upwind flux in the
case of the scalar advection equation) applied to hyperbolic systems
is always stable and $(p+1)$-th order accurate. However, Godunov's
theorem states that linear monotone\footnote{A monotone scheme is a
  scheme that does not generate new extrema.} schemes are at most
first-order accurate. Hence, for $p>0$, we expect the DG method to
generate new extrema, which are observed as undershoots and overshoots
for the scalar advection equation. However, the DG method does have
the additional property that if the DG solution fields\footnote{For a
  system of equations this refers to the characteristic variables
  obtained from the diagonalisation of the hyperbolic system.} are
bounded element-wise, \emph{i.e.} at each element face a solution
field lies between the average value for that element and the average
value for the neighbouring element on the other side of the face, then
the element-averaged DG field (\emph{i.e.} the projection of the DG
field to \Pzero) does not obtain any new minima. This result only holds if
the explicit Euler timestepping method (or one of the higher-order
extensions, namely the Strongly Structure Preserving Runge-Kutta
(SSPRK) methods) is used. Hence, the DG field can be made monotonic by
adjusting the solution at the end of each timestep (or after each
SSPRK stage) so that it becomes bounded element-wise. This is done in
a conservative manner \emph{i.e.} without changing the element-averaged
values. For \Pone, only the slopes can be adjusted to make the solution
bounded element-wise, and hence the adjustment schemes are called
\emph{slope limiters}.

\paragraph{Types of slope limiter}
\label{Sect:ND_DG_slope_limiters}
\index{slope limiters}
There are two stages in any slope limiter. First all the elements
which do not currently satisfy the element bounded condition must be
identified. Secondly, the slopes (and possibly higher-order components
of the solution) in each of these elements must be adjusted so that
they satisfy the bounded condition. In general, this type of
adjustment has the effect of introducing extra diffusion, and so it is important to ($a$)
identify as few elements as possible, and ($b$) adjust the slopes as
little as possible, in order to introduce as little extra diffusion as
possible. For high-order elements, exactly how to do this is a very
contentious issue but a few approaches are satisfactory for low-order
elements.
\subparagraph{Cockburn-Shu limiter}
This limiter, introduced in \cite{cockburn2001}, only checks the element
bounded condition at face centres. There is a tolerance parameter, the
TVB factor, which controls how sensitive the method is to the bounds
(the value recommended in the paper is 5) and a limit factor, which
scales the reconstructed slope (the value recommended in the paper is
1.1). The method seems not to be independent of scaling, and the paper
assumes an $\mathcal{O}(1)$ field, so these factors need tuning for
other scalings.

\subparagraph{Vertex-based limiter}
This limiter, introduced in \cite{Ku2010}, works on a hierarchical
Taylor expansion. It is only currently implemented for linear elements.
In this case, the DG field $c$ in one element $e$ may be written as
\[
c = \bar{c} + c', \quad \bar{c} = \frac{\int_e c dV}{Vol(e)},
\]
and the limiter replaces $c$ with $c_L$ given by
\[
c = \bar{c} + \alpha c',
\]
finding the maximum value $\alpha>0$ such that at each vertex, $c$ is
bounded by the maximum and minimum of the values of T at that vertex
over all elements that share the vertex.  This limiter has no
parameters, and is the currently recommended limiter.

\subparagraph{Hermite-WENO limiter} \label{sect:ND_hermite_weno_limiter}
This limiter makes use of the Weighted Essentially Non-Oscillatory
(WENO) interpolation methods, originally used to obtain high-order
non-oscillatory fluxes for finite volume methods, to reconstruct the
solution in elements which do not satisfy the element-wise bounded
condition (sometimes referred to as ``troubled elements''). The
principle is the following: if we try to reconstruct the solution as a
$p$-th order polynomial in an element by fitting to the cell average
of the element and of some neighbouring elements, then if there is a
discontinuity in the solution within the elements used then the $p$-th
order polynomial is very wiggly and will exceed its bounds. The WENO
approach is as follows:
\begin{enumerate}
\item Construct a number of polynomials approximations using various
  different combinations of elements, each having the same cell-average
  in the troubled element.
\item Calculate a measure of the wiggly-ness of each polynomial (called
an oscillation indicator).
\item The reconstructed solution is a weighted average of each of the
  polynomials, with the weights decreasing with increasing oscillation
  indicator.
\end{enumerate}
Thus if there is a discontinuity to one side of the element, the
reconstructed solution will mostly use information from the other side
of the discontinuity. The power law relating the weights with the
oscillation indicators can be selected by the user, but is configured
so that in the case of very smooth polynomials, the reconstruction
accuracy exceeds the order of the polynomials \emph{e.g.} 5th order for
3rd order polynomials.

In practise, making high order reconstructions from unstructured
meshes is complicated since many neighbouring elements must be
used. If one also uses the gradients from the element and the direct
neighbours (Hermite interpolation) this is sufficient to obtain an
essentially non-oscillatory scheme. This is called the Hermite-WENO
method. In this method applied to \Pone, the complete set of
approximations used for the solution in an element are:
\begin{itemize}
\item The original solution in the element $E$.
\item Solutions with gradient constructed from the mean value of the
  element $E$ and $d$ other elements which share a face with $E$.  In
  2D this is 3 solutions, and in 3D this is 4 solutions. Each of these
  solutions has the same mean value as the original solution.
\item Solutions with gradient the same as one of the $d$ neighbouring
  elements. Each of these solutions has the same mean value as the
  original solution.
\end{itemize}
This is a total of $2d+1$ solutions which must be weighted according
to their oscillator indicator value.

The advantage of using WENO (and H-WENO) reconstruction is that it
preserves the order of the approximation, and hence it is not quite so
important to avoid it being used in smooth regions (other limiters
would introduce too much diffusion in those regions). However,
reconstruction is numerically intensive and so to make H-WENO more
computationally feasible, it must be combined with a discontinuity
detector which identifies troubled cells. It does not do too much
damage to the solution if the discontinuity detector is too strict
\emph{i.e.}  identifies too many elements as troubled, but will reduce
the efficiency of the method.

\subsubsection{Diffusion term for DG}
\label{sec:NM_DG_diffusion}
\index{Galerkin!discontinuous!diffusion}
\index{diffusion!discontinuous Galerkin} In this section we describe
the discretisation of the diffusion operator using discontinuous
Galerkin methods. We concentrate on solving the equation
\begin{equation}
\label{eq:poisson}
\nabla^2 c = f,
\end{equation}
although this can easily be extended to the advection-diffusion and
momentum equations by replacing $f$ with the rest of the equation.
Discretising the Poisson equation \eqref{eq:poisson} using
discontinuous Galerkin is a challenge since discontinuous fields are
not immediately amenable to introducing second-order operators (they
are best at advection): the treatment of the diffusion operator is one
of the main drawbacks with discontinuous Galerkin methods.  The
standard continuous finite element approach is to multiply equation
\eqref{eq:poisson} by a test function and integrate the Laplacian by
parts, leading to integrals containing the gradient of the trial and
test functions. In DG methods, since the trial and test functions contain
discontinuities, these integrals are not defined. There are two
approaches to circumventing this problem which have been shown to be
essentially equivalent in \cite{arnold2002}, which we describe in this
section.

The first approach, which leads to the class of interior penalty
methods, is to integrate the Laplacian by parts separately in each
element (within which the functions are continuous), and the equations
become
\begin{equation}
\label{eq:poisson-parts}
\sum_e\left(-\int_e\nabla \phi^\delta\cdot\nabla c^\delta +
 \int_{\partial e} \phi^\delta \vec{n}\cdot\nabla c^\delta\right) =
\sum_e\int_e \phi^\delta f^{\delta},
\end{equation}
where $e$ is the element index $\int_e$ indicates an integral over
element $e$, $\int_{\partial e}$ indicates an integral over the
boundary of $e$, $\phi^\delta$ is the DG test function, $c^\delta$ is
the DG trial function, and $f^\delta$ is the DG approximation of $f$.
The next step is to notice that for each facet (face in 3D, edge in 2D
or vertex in 1D) there is a surface integral on each side of the
facet, and so equation \eqref{eq:poisson-parts} becomes
\begin{equation}
\label{eq:poisson-parts-facets}
-\sum_e\int_e\nabla \phi^\delta\cdot\nabla c^\delta +
\sum_\Gamma\int_{\Gamma} [[\nabla c^\delta{\phi}^\delta]] =
\sum_e\int_e \phi^\delta f^{\delta},
\end{equation}
where $\Gamma$ is the facet index, $\int_{\Gamma}$ indicates an
integral over facet $\Gamma$, and the jump bracket
$[[\vec{v}^\delta]]$ measures the jump in the normal component of
$\vec{v}^\delta$ across $\Gamma$ and is defined by
\[
[[\vec{v}^\delta]] = \vec{v}^\delta|_{e^+}\cdot\vec{n}^+ +
\vec{v}^\delta|_{e^-}\cdot\vec{n}^-,
\]
where $e^+$ and $e^-$ are the elements on either side of facet
$\Gamma$, $\vec{v}^\delta_{e^{\pm}}$ is the value of the vector-valued
DG field $\vec{v}^\delta$ on the $e^{\pm}$ side of $\Gamma$, and
$\vec{n}^{\pm}$ is the normal to $\Gamma$ pointing out of
$E^{\pm}$. The problem with this formulation is that there is still no
communication between elements, and so the equation is not
invertible. The approximation is made consistent by making three
changes to equation \eqref{eq:poisson-parts-facets}. Firstly, in the
facet integral, the test function $\phi^\delta$ (which takes different
values either side of the face) is replaced by the average value,
$\{\phi^\delta\}$ defined by
\[
\{\phi^\delta\} = \frac{\phi^\delta|_{E^+} + \phi^\delta|_{E^-}}{2},
\]
leading to
\begin{equation}
\label{eq:poisson-parts-facets-average}
-\sum_e\int_e\nabla \phi^\delta\cdot\nabla c^\delta +
\sum_\Gamma\int_{\Gamma} [[\nabla c^\delta]]\{\phi^\delta\} =
\sum_e\int_e \phi^\delta f^{\delta}.
\end{equation}
Secondly, to make the operator symmetric (required for adjoint
consistency, also means that the conjugate gradient method can be used
to invert the matrix), an extra jump term is added, leading to
\begin{equation}
\label{eq:poisson-parts-facets-average-symmetric}
-\sum_e\int_e\nabla \phi^\delta\cdot\nabla c^\delta +
\sum_\Gamma\int_{\Gamma} [[\nabla c^\delta]]\{\phi^\delta\} +
\{c^\delta\}
[[\nabla\phi^\delta]] = \sum_e\int_e \phi^\delta f^{\delta}.
\end{equation}
Note that this symmetric averaging couples together each node in
element $e$ with all the nodes in the elements which share facets with
element $e$. Thirdly, a penalty term is added which tries to reduce
discontinuities in the solution, and the discretised equation becomes
\begin{equation}
\label{eq:poisson-penalty}
-\sum_e\int_e\nabla \phi^\delta\cdot\nabla c^\delta +
\sum_\Gamma\int_{\Gamma} [[\nabla c^\delta]]\{\phi^\delta\} +
\{c^\delta\}
[[\nabla\phi^\delta]] + \alpha(\phi^\delta,c^\delta)
= \sum_e\int_e \phi^\delta f^{\delta},
\end{equation}
where $\alpha(\cdot,\cdot)$ is the penalty functional which satisfies
the following properties:
\begin{enumerate}
\item Symmetry:
  $\alpha(c^\delta,\phi^\delta)=\alpha(\phi^\delta,c^\delta)$.
\item Positive semi-definiteness: $\alpha(c^\delta,c^\delta)\geq 0$.
\item Continuity-vanishing: $\alpha(c^\delta,c^\delta)=0$ when $c^\delta$ is continuous
 across all facets.
\item Discontinuity-detecting: $\alpha(c^\delta,c^\delta)$ increases
  as the discontinuities across facets increase.
\end{enumerate}
The form of equation \eqref{eq:poisson-penalty} is called the
\emph{primal form}.  Note that, due to the continuity-vanishing
property, equation \eqref{eq:poisson-penalty} is satisfied by the
exact solution (which is always continuous) to the Poisson equation,
which is the required \emph{consistency condition}.  In defining the
particular form the penalty functional $\alpha$ it is necessary to
maintain a balance: if the functional penalises discontinuities too
much then the resulting matrix is ill-conditioned, if it penalises
discontinuities too little then there is not enough communication
between elements and the numerical solution does not converge to the
exact solution. The particular form of $\alpha$ for the Interior
Penalty method is described below.

The second approach (the Local Discontinuous Galerkin (LDG) framework
\citep{cockburn1998,sherwin2006} which leads to Bassi-Rebay and
Compact Discontinous Galerkin methods) is to introduce a vector field
$\vec{\xi}$, to rewrite the Poisson equation as a system of
first-order equations
\begin{equation}
\label{eq:poisson-first-order}
\nabla\cdot\vec{\xi} = f, \quad \vec{\xi}=\nabla c,
\end{equation}
and to finally eliminate the vector field $\vec{\xi}$. This
elimination is possible to do locally (\emph{i.e.} only depending on
the values of $c$ in nearby elements) since the mass matrix for DG
fields is block diagonal and so the inverse mass matrix does not
couple together nodes from different elements. For discontinuous
Galerkin methods, we introduce a discontinuous vector test function
$\vec{w}^{\delta}$. Multiplication of equations
\eqref{eq:poisson-first-order} by test functions, integrating over a
single element $E$ and applying integration by parts leads to
\begin{equation}
\label{eq:ldg}
-\int_e\nabla\phi^{\delta}\cdot\vec{\xi}^\delta + \int_{\partial e}
\phi^\delta\vec{n}\cdot\hat{\vec{\xi}}^{\delta} =
\int_e\phi^\delta f^\delta,
\quad
-\int_e\nabla\cdot\vec{w}^\delta c^\delta + \int_{\partial e}
\vec{n}\cdot\vec{w}^\delta \hat{c}^\delta = \int_e\vec{w}^\delta\cdot\vec{\xi}^\delta.
\end{equation}
This form of the equations is called the \emph{dual form}.  The exact
definition of the particular scheme depends on how the surface values
(fluxes) $\hat{\vec{\xi}}^\delta$ and $\hat{c}^\delta$ are
defined. The choice of these fluxes has an impact on the stability
(whether there are any spurious modes), consistency (whether the
discrete equation is satisfied by the exact solution), convergence
(whether and how fast the numerical solution approaches the exact
solution), and sparsity (how many non-zero elements there are in the
resulting matrix). It is worth noting at this point that the method of
rewriting the second-order operator as a first-order system has some
superficial connections with the discrete pressure projection method
for continuous finite element methods as described in Section
\ref{Sect:pressure_correction}. However, many of the ideas do not
carry over to the discontinuous Galerkin framework, for example, it is
neither necessary nor sufficient to reduce the polynomial order of
$\phi^\delta$ relative to the polynomial order of
$\vec{\xi}^\delta$. The issues of stability, consistency, convergence
and sparsity for DG discretisations of the diffusion operator are
extremely subtle and there is an enormous literature on this topic; it
remains a dangerous tar pit for the unwary developer looking to invent
a new DG diffusion operator discretisation.

It was shown in \citet{arnold2002} (which is an excellent review of DG
methods for elliptic problems) that numerical schemes obtained from
this second approach can be transformed to primal form, resulting
precisely in discretisations of the form \eqref{eq:poisson-penalty}
with some particular choice of the functional $\alpha$. Hence, it is
possible to describe the three options available in Fluidity together,
in the following subsections.

\paragraph{Interior Penalty}
The Interior Penalty method is a very simple scheme with penalty
functional of the form
\begin{equation}
\label{eq:ip}
\alpha(\phi^\delta,c^\delta) = \sum_{\Gamma}C_{\Gamma}\int_{\Gamma}
[[\phi^\delta]]\cdot[[c^\delta]],
\end{equation}
where for a scalar function $\phi^\delta$, the jump bracket
$[[\cdot]]$ is a vector quantity defined as
\[
[[\phi^\delta]] = \phi^\delta|_{E^+}\vec{n}^+ + \phi^\delta|_{E^-}\vec{n}^-.
\]
For convergence, the constant $C_{\Gamma}$ should be positive and
proportional to $h^{-1}$, where $h$ is the element edge length. This
needs to be carefully defined for anistropic meshes.

\paragraph{Bassi-Rebay}
\label{BassiRebay}
\index{diffusion!discontinuous Galerkin!Bassi-Rebay}
\index{Bassi-Rebay}
The scheme of Bassi-Rebay \citep{bassi1997} is in some sense the most
simple choice within the LDG framework, in which the fluxes are just
taken from the symmetric averages:
\[
\hat{\xi}^\delta = \{\vec{\xi}^\delta\}, \quad
\hat{\phi}^\delta = \{\phi^\delta\}.
\]
This scheme was analysed in \cite{arnold2002}, and was shown to only
converge in the following rather weak sense: if the numerical solution
and exact solution are projected to piecewise constant
(\Pzero) functions then these projected solutions converge to each at order $p+1$,
without this projection they only converge at order $p$, where $p$ is
the polynomial order used in the DG element. Furthermore, the
Bassi-Rebay scheme has a very large stencil (large number of non-zero
values in the resulting matrix). For this reason, other more
sophisticated flux choices have been investigated.

\paragraph{Compact Discontinuous Galerkin}
\label{CDG}
\index{Compact Discontinuous Galerkin}
\index{diffusion!discontinuous Galerkin!CDG}

The Compact Discontinuous Galerkin (CDG) scheme
\cite{peraire2008} has a rather complex choice of fluxes based on ``lifting
operators'' (see the paper for more information). When transforming the
equations to primal form, this choice of flux results in a sophisticated
penalty function with two terms. The first term exactly cancels part of the
flux integrals in equation \eqref{eq:poisson-penalty} so that all of the
symmetric fluxes using the averaging bracket $\{\cdot\}$ are replaced by the
flux evaluated on one particular (arbitrarily chosen) side of the facet
(there are various schemes for making this choice). The second term only
couples together nodes which share the same facet. Both of these terms
result in a much smaller stencil than for the Bassi-Rebay scheme in
particular and other LDG schemes in general. Furthermore, the scheme is
observed to be stable, consistent, and optimally convergent (numerical
solutions converge at order $(p+1)$). The scheme optionally includes the
interior penalty term from equation \eqref{eq:ip}, but the constant may be
independent of $h$. This term only appears to be necessary for the
mathematical proofs of stability and convergence since in practise good
results are obtained without this term (in fact the results are usually more
accurate without this term). Since the penalty term has only one tunable
constant (which may be set to zero) with does not depend on $h$, this makes
the CDG scheme very attractive for anisotropic elements, including large
aspect ratio meshes such as those used in large scale ocean modelling.


\subsection{Control volume discretisation}
\label{ControlVolumeAdvection}
\index{control volume!advection}
\index{advection-diffusion equation!control volume}

Finite volume discretisations may be thought of as the lowest order discontinuous Galerkin method, using piecewise constant shape functions (see Figure \ref{fig:p0shapefunctions}).  In \fluidity\ this type of element centred discretisation is handled through the discontinuous Galerkin method, however the model also supports an alternative finite volume discretisation referred to as a control volume (CV) discretisation.

\begin{figure}[tbp]
\begin{center}
\begin{tabular}{c}
\xfig{numerical_discretisation_images/P0shapefunction1d}  \\
\xfig{numerical_discretisation_images/P0shapefunction2d} 
\end{tabular}
\caption{One-dimensional (a) and two-dimensional (b) schematics of piecewise constant, element centred shape functions.  The shape function has value $1$ at node $A$ and across the element, $e$, descending to $0$ at the element boundaries.  As with other discontinuous shape functions, the support, $s$, coincides with the element, $e$.}
\label{fig:p0shapefunctions}
\end{center}
\end{figure}

The control volume discretisation uses a dual mesh constructed around the nodes of the parent finite element mesh.  In two dimensions this is constructed by connecting the element centroids to the edge midpoints while in three dimensions the face centroids are also introduced.  For cube meshes (quadrilaterals in 2D and hexahedra in 3D) this produces a staggered mesh of the same type.  For simplex meshes this process produces complex polyhedra (see Figures \ref{fig:cornerunstruct} and \ref{fig:cvmesh3d}).

\begin{figure}[tbp]
\begin{center}
\begin{tabular}{lr}
\xfig{numerical_discretisation_images/corner_unstructured} & \xfig{numerical_discretisation_images/corner_unstructured_cv}
\end{tabular}
\caption{Comparison between (a) a two-dimensional finite volume simplex mesh and (b) the equivalent control volume dual mesh (solid lines) constructed around a piecewise linear continuous finite element parent mesh (dashed lines).  In the finite volume mesh the nodes (e.g. A) are element centred whereas in the control volume dual mesh the nodes are vertex based.  In 2D the control volumes are constructed around A by connecting the centroids of the neighbouring triangles to the edge midpoints.  See Figure \ref{fig:cvmesh3d} for the equivalent three-dimensional construction.}
\label{fig:cornerunstruct}
\end{center}
\end{figure}

\begin{figure}[tb]
\begin{center}
\xfig{numerical_discretisation_images/P1controlvolume3d}
\caption{The six dual control volume mesh faces within a piecewise linear tetrahedral parent mesh element.  Each face is constructed by connecting the element centroid, the face centroids and the edge midpoint.}
\label{fig:cvmesh3d}
\end{center}
\end{figure}

Once the dual control volume mesh has been defined, it is possible to discretise the advection-diffusion equation \eqref{eq:weak_adv_diff} using piecewise constant shape functions within each volume, $v$ (see Figure \ref{fig:cvshapefunctions}).  However, as with the discontinuous Galerkin method the equation is only well defined when integrated by parts within such a volume, $v$, allowing us to write:
\begin{equation}\label{eq:adv_diff_cv}
  \int_v \frac{\partial c}{\partial t} +
    \int_{\partial v} \widehat{\vec{n}\cdot\vec{u}~c} -
    \widehat{\vec{n}\cdot\tensor{\mu}\cdot\grad c}
    = 0,
\end{equation}
Note that the test function present in the previous discretisation sections, $\phi$, can be dropped from the equation as it is $1$ everywhere within the volume $v$.  Furthermore, terms involving the gradient of either $\phi$ or $c$ can be dropped as both as constant functions.  The boundary integral for the viscosity $\tensor{\mu}$ is a special case that will be dealt with in a later section.

\begin{figure}[btp]
\begin{center}
\begin{tabular}{lr}
\xfig{numerical_discretisation_images/P1cvshapefunction1d} & \xfig{numerical_discretisation_images/P2cvshapefunction1d} \\
\xfig{numerical_discretisation_images/P1cvshapefunction2d} & \xfig{numerical_discretisation_images/P2cvshapefunction2d}
\end{tabular}
\caption{One-dimensional (a, b) and two-dimensional (c, d) schematics of piecewise constant control volume shape functions and dual meshes based on the parent (dashed lines) linear (a, c) and quadratic (b, d) finite element meshes.  The shape function has value $1$ at node $A$ descending to $0$ at the control volume boundaries.  The support, $s$, coincides with the volume, $v$.}
\label{fig:cvshapefunctions}
\end{center}
\end{figure}

As with the discontinuous Galerkin discretisation, the hatted terms represent fluxes across the volume facets: and therefore from one volume to the other.  Due to the discontinuous nature of the fields, there is no unique value for these flux terms, however the requirement that $c$ by a conserved quantity does demand that adjacent volumes make a consistent choice for the flux between them.  The choice of flux schemes tehrefore forms a critical component of the control volume method.

The application of boundary conditions occurs in the same manner as for the discontinuous Galerkin method.  The complete system of equations is formed by summing over all the volumes.  Assuming weakly applied boundary conditions, this results in:
\begin{multline}
  \sum_v \left\{ \int_v \frac{\partial c}{\partial t}
  + \int_{\partial v\,\cap\dOmega^{D_c}_{-}} \vec{n}\cdot\vec{u}~g_D
  + \int_{\partial v\,\cap\dOmega^{D_c}_{+}} \vec{n}\cdot\vec{u}~c
  - \int_{\partial v\,\cap\dOmega^{D_c}} \vec{n}\cdot\tensor{\mu}\cdot\grad c \right. \\
  \left. + \int_{\partial v\,\cap\dOmega^{N_c}} \vec{n}\cdot\vec{u}~c
  - \vec{n}\cdot\tensor{\mu}\cdot\grad g_N 
  + \int_{\partial v\,\setminus\dOmega} \widehat{\vec{n}\cdot\vec{u}~c}
  - \widehat{\vec{n}\cdot\tensor{\mu}\cdot\grad c} \right\}
    = 0.
\end{multline}

\subsubsection{Control Volume advection}
\label{Sect:ND_control_volume_advection}

Consider first the case in which $\tensor{\mu}=0$. In this case, equation \eqref{eq:adv_diff_cv}\ reduces to:
\begin{equation}\label{eq:adv_cv}
  \int_v \frac{\partial c}{\partial t} +
    \int_{\partial v} \widehat{\vec{n}\cdot\vec{u}~c}
    = 0,
\end{equation}
and the question becomes, how do we represent the flux $\widehat{\vec{n}\cdot\vec{u}~c}$\ ?

\fluidity\ supports multiple different advective fluxes for CV.  Unlike with DG the advection velocity $\vec{u}$ is always well-defined at the control volume facets owing to the fact that they cross through the centre of the elements of the parent mesh where the velocity is continuous.  Therefore it is only necessary to describe how the face value of $c$ is going to be defined.

In the following paragraphs we will refer to the donor or central value, $c_{c_k}$, and the downwind value, $c_{d_k}$.  These are associated with facet $k$ of the control volume dual mesh and are defined depending on the direction of the flux across that facet.  If the flow across facet $k$ is exiting volume $v$, i.e. $\vec{n}\cdot\vec{u}|_k>0$, then the donor value, $c_{c_k}$ is the value in the volume, $c_v$ and the downwind value, $c_{d_k}$ is the value in the volume that the flux is entering.  Similarly if the flow is entering the volume $v$, i.e. $\vec{n}\cdot\vec{u}|_k<0$, then the donor value, $c_{c_k}$, is the value from the neighbouring control volume while the downwind value, $c_{d_k}$, is the value in the volume, $c_v$.  By default only first order quadrature is performed on the control volume facets, however if higher order control volume facet quadrature is selected then $k$ refers to each quadrature point on the facet.

\paragraph{First Order Upwinding} \label{sec:cv_fou}

In this case, the value of $c$ at each quadrature point on each facet is taken to be the donor value, $c_{c_k}$.  Then \eqref{eq:adv_cv} becomes:
\begin{equation}
  \int_v \ppt{c} +
    \sum_k\int_{\partial v_k} \vec{n}\cdot\vec{u}~c_{c_k}
    = 0.
\end{equation}
First order upwinding is stable in the sense of boundedness, assuming an appropriate temporal discretisation is selected, it is however very diffusive so normally a higher order but less stable face value scheme is selected.

\paragraph{Trapezoidal} \label{sec:trap}

In this case, the average of the donor and downwind values, $\frac{c_{c_k}+c_{d_k}}{2}$, is taken as the value of $c$ at each facet.  This is generally unstable regardless of the temporal discretisation chosen so requires limiting (see below).

\paragraph{Finite Element Interpolation} \label{sec:cvfe}

In this case, the value of $c$ at each quadrature point of the facet is interpolated using the finite element basis functions on the parent mesh.  This is possible as the nodes of both the dual and parent meshes are co-located.  Like the trapezoidal face value this method is also generally unstable so normally requires limiting (see below).

\paragraph{First Order Downwinding} \label{sec:fod}

In this case, the value of $c$ at each quadrature point of the facet is set to the downwind value, $c_{d_k}$.  First order downwinding is unconditionally unstable and is intended for demonstration purposes only.

\paragraph{Face Value Limiting} \label{sec:cvlimiting}

As noted above, several of the face value schemes above result in unstable, unbounded advective fluxes.  To compensate for this it is possible to limit the face value in an attempt to maintain boundedness.  This requires an estimate of the upwind flux entering the control volume so we introduce a new value, $c_{u_k}$, for the value upwind of the donor volume, $c_{c_k}$.  On a fully unstructured mesh this value is not directly available so instead it must be estimated.  For cube meshes the default behaviour is to estimate the upwind value from the minimum or maximum of the surrounding nodes depending on the gradient between the donor and downwind values.  On simplex meshes \fluidity\ defaults to more accurate schemes that project the value at the upwind position based on all the nodes in the parent element immediately upwind of the donor node (see Figure \ref{fig:unstructupwindnode}(a)).  As this value is not necessarily bounded itself it is also possible to bound it so that it falls within the values of the upwind element.  Additionally, on a boundary it is possible to reflect the upwind value back into the domain (see Figure \ref{fig:unstructupwindnode}(b)), which may be more appropriate.

\begin{figure}[tbp]
\begin{center}
\begin{tabular}{lr}
\xfig{numerical_discretisation_images/upwind_node_internal} & \xfig{numerical_discretisation_images/upwind_node_boundary}
\end{tabular}
\caption{Calculation of the upwind value, $c_{u_k}$, on an unstructured simplex mesh (a) internally and (b) on a boundary.  The control volume mesh is shown (solid lines) around the nodes (black circles) which are co-located with the solution nodes of the parent piecewise linear continuous finite element mesh (dashed lines).  An initial estimate of the upwind value, $c^*_{u_k}$, is found by interpolation within the upwind parent element.  The point-wise gradient between this estimate and the donor node, $c_{c_k}$, is then extrapolated the same distance between the donor and downwind, $c_{d_k}$, to the upwind value, $c_{u_k}$.}
\label{fig:unstructupwindnode}
\end{center}
\end{figure}

Once an upwind value, $c_{u_k}$,  is available it is possible to estimate whether the chosen face value, $c_{f_k}$, will cause the solution to become unbounded.  \fluidity\ uses a normalised variable diagram \citep[NVD, ][]{waterson_design_2007, wilson_phdthesis_2009}  based scheme to attempt to enforce a total variation diminishing \citep[TVD, ][]{leveque_finite-volume_2002} definition of boundedness.  First, a normalised donor value:
\begin{equation}
\bar{c}_{c_k} = \frac{c_{c_k}-c_{u_k}}{c_{d_k}-c_{u_k}}
\end{equation}
and a normalised face value:
\begin{equation}
\bar{c}_{f_k} = \frac{c_{f_k}-c_{u_k}}{c_{d_k}-c_{u_k}}
\end{equation}
are defined.  These are then used to define the axes of the NVD.  This has the advantage of dealing with multiple configurations in a single coordinate system \citep[NVD, ][]{waterson_design_2007, wilson_phdthesis_2009}.

Many face value limiting schemes can be implemented on the NVD \citep{leonard_ultimate_1991}.  \fluidity\ provides two options: the Sweby \citep{sweby_high_1984} and ULTIMATE \citep{leonard_ultimate_1991} limiters (see Figure \ref{fig:limiters}).

\begin{figure}[tbp]
\begin{center}
\xfig{numerical_discretisation_images/swebynvd} \\ \vspace{0.5cm}
\xfig{numerical_discretisation_images/ultimatenvd}
\caption{The Sweby (a) and ULTIMATE (b) limiters (shaded regions) represented on a normalised variable diagram (NVD).  For comparison the trapezoidal face value scheme is plotted as a dashed line in both diagrams.}
\label{fig:limiters}
\end{center}
\end{figure}

\subparagraph{Sweby Limiter}

The Sweby limiter \citep{sweby_high_1984} defines a region on the NVD (shaded grey in Figure \ref{fig:limiters}(a)) that is considered bounded.  Any combination of normalised face and donor values falling within this region is left unchanged.  Values falling outside this area are `limited' along lines of constant normalised donor value back onto the top or bottom of the stable region, or if they fall outside the range $0<\bar{c}_{c_k}<1$, back to first order upwinding (represented by the diagonal line on a NVD).

As an example the trapezoidal face value scheme is plotted as a dashed line in Figure \ref{fig:limiters}(a).  It only crosses the Sweby limiter region for a small range of normalised donor values and it is only across this range that the face value will not be limited.  Outside of this range the trapezoidal face value scheme is considered unstable and must be limited.  This is akin to saying that the trapezoidal face value scheme may only be used in regions where the solution is sufficiently smooth.  In particular, outside the range $0<\bar{c}_{c_k}<1$, the solution is already locally unbounded (in a TVD sense) and hence only first order upwinding may be used to restore local boundedness.

\subparagraph{ULTIMATE Limiter}

An alternative definition of the sufficiently smooth region of the NVD is provided by the ULTIMATE limiter \citep[see Figure \ref{fig:limiters}(b), ][]{leonard_ultimate_1991}.  This works using the same principals as the Sweby limiter however the region defined as being stable now incorporates an upper bound that depends on the Courant number at the control volume face to the donor value, $\gamma_{c_f}$.  Hence the region expands and contracts depending on the Courant number, collapsing back entirely to first order upwinding once a Courant number of one is attained.  It is therefore principally designed for explicit advection.  In fact for explicit advection, with the correct control volume based definition of the Courant number, this limiter exactly defines the total variation bounded region, guaranteeing boundedness in that sense \citep{leonard_ultimate_1991, desprs_contact_2001}.

\paragraph{HyperC} \label{sec:hyperc}

HyperC uses the same principals as the ULTIMATE limiter but instead of limiting other face value schemes using it, HyperC uses the NVD as a face value scheme directly \citep{leonard_ultimate_1991}.  Given a donor, downwind and upwind value (and hence a normalised donor value) it simply uses the upper boundary of the TVD zone to calculate the normalised face value (and hence a face value, see Figure \ref{fig:hyperc}).  For explicit advection, with the correct definition of the control volume based Courant number,  $\gamma_{c_f}$, this scheme aims to produces a minimally diffusive advection scheme.  It is however, only intended for the advection of step-like functions as it results in distortion and staircasing of smoother functions. \citet{wilson_phdthesis_2009} discusses the implementation of HyperC in \fluidity\ and its extension to multiple dimensions.

\begin{figure}[tbp]
\begin{center}
\xfig{numerical_discretisation_images/hypercnvd}
\caption{The HyperC face value scheme represented on a normalised variable diagram (NVD).}
\label{fig:hyperc}
\end{center}
\end{figure}

\paragraph{UltraC} \label{sec:ultrac}

Like HyperC, UltraC uses the NVD to define a face value directly.  The difference is that is uses a total variation bounded (TVB) rather than a total variation diminishing (TVD) definition of boundedness.  As with HyperC this aims to minimise numerical diffusion while advecting step-like functions but will distort smoother fields.

\begin{figure}[tbp]
\begin{center}
\xfig{numerical_discretisation_images/ultracnvd} \\ \vspace{0.5cm}
\xfig{numerical_discretisation_images/ultracmodnvd}
\caption{The UltraC face value scheme represented on (a) a normalised variable diagram (NVD) and (b) a modified normalised variable diagram.  For comparison (a) also shows the HyperC face value scheme as a dotted line.}
\label{fig:ultrac}
\end{center}
\end{figure}

UltraC can be represented on an NVD (see Figure \ref{fig:ultrac}(a)), where it is obvious that the bounded range has been extended beyond TVD range used by HyperC, $0<\bar{c}_{c_k}<1$ (shown as a dotted line in Figure \ref{fig:ultrac}(a)).  Instead it is now dependent on two new values: a target upwind value, $c_{tu}$, and a target downwind value, $c_{td}$ and their equivalent normalised versions:
\begin{equation}
\bar{c}_{tu_k} = \frac{c_{tu_k}-c_{u_k}}{c_{d_k}-c_{u_k}}
\end{equation}
and a normalised face value:
\begin{equation}
\bar{c}_{td_k} = \frac{c_{td_k}-c_{u_k}}{c_{d_k}-c_{u_k}}
\end{equation}
respectively.  These are defined based on user prescribed target maximum and target minimum values depending on the local slope of the solution.  These maximum and minimum values describe the range of values in which any solution is bounded.

The target upwind and downwind values can also be used to modify the definition of the normalised donor and face values:
\begin{equation}
\check{c}_{c_k} = \frac{c_{c_k}-c_{tu_k}}{c_{td_k}-c_{tu_k}}
\end{equation}
and a normalised face value:
\begin{equation}
\check{c}_{f_k} = \frac{c_{f_k}-c_{tu_k}}{c_{td_k}-c_{tu_k}}
\end{equation}
which allows a modified normalised variable diagram to be defined.  In this case UltraC looks superficially like HyperC (see Figure \ref{fig:ultrac}(b)).

For more information on the development of UltraC in \fluidity\ see \citet{wilson_phdthesis_2009}.

\paragraph{PotentialUltraC} \label{sec:potultrac}

One side effect of using a total variation bounded scheme, as in UltraC, is that small isolated regions of field that are already easily within the minimum and maximum bounds are advected at spurious velocities.  This can be solved by either switching to HyperC or modifying the flux in the vicinity of these regions.  PotentialUltraC implements these options.  More details of this scheme can be found in \citet{wilson_phdthesis_2009}.

\paragraph{Coupled Limiting} \label{sec:coupledlimiter}

Under some circumstances it becomes necessary to limit the face values of a field based not only on the boundedness of the field itself but based on the boundedness of other fields as well.  This is implemented in \fluidity\ as a coupled control volume spatial discretisation.  This allows the user to select any of the above face value schemes, which are then limited to ensure that both the field and the sum of the field with the other related fields is in some sense bounded.  This is particularly useful for multiple material simulations where the volume fractions must not only remain individually bounded between $0$ and $1$ but their sum must also be similarly bounded.

Limiting based on a summation of fields requires the user to specify a priority ordering describing the order in which the individual fields should be advected and summed.  Hence we now consider a field $c^I$, where $I$ indicates the priority of the field from $1$ (highest priority) to the number of related fields, $N$ (lowest priority).  We then introduce the variables:
\begin{equation}
c^{\sum_{I}} = \sum_{i=1}^I c^i
\end{equation}
and
\begin{equation}
c^{\sum_{I-1}} = \sum_{i=1}^{I-1} c^i
\end{equation}
along with their related normalised versions.  Using these variable and the total variation bounded (TVB) definition of boundedness introduced in UltraC it is possible to check whether the current field face value, $c_{f}^I$ (and its normalised equivalent $\bar{c}_{f}^I$), falls within a region on the normalised variable diagram in which the summation up to the current field, $c^{\sum_{I}}$, is itself bounded, in a TVB sense (see Figure \ref{fig:coupledlimiter}).

\begin{figure}[tbp]
\begin{center}
\xfig{numerical_discretisation_images/coupledlimiter}
\caption{The coupled limiter for the field $c^I$ represented by the grey shaded area on a normalised variable diagram (NVD).  Labels in the upper left blue region refer to the case when the difference between the parent sum downwind and upwind values has the same sign as the limited field, $\text{sign}\left(c^{\sum_{I}}_{d_k}-c^{\sum_{I}}_{u_k}\right) = \text{sign}\left(c^{I}_{d_k}-c^{I}_{u_k}\right)$.  Similarly, labels in the lower right yellow region refer to the case when the signs of the slopes are opposite, $\text{sign}\left(c^{\sum_{I}}_{d_k}-c^{\sum_{I}}_{u_k}\right) \neq \text{sign}\left(c^{I}_{d_k}-c^{I}_{u_k}\right)$. The regions are separated by the upwinding line, $\bar{\hat{c}}^{I}_{f} = \bar{c}^{I}_{c_k} + \bar{c}^{\sum_{I-1}}_{c_k} - \bar{c}^{\sum_{I-1}}_{f}$.}
\label{fig:coupledlimiter}
\end{center}
\end{figure}

The coupled limiter described in Figure \ref{fig:coupledlimiter} uses information from the previous summation of fields, $c^{\sum_{I-1}}$, to asses whether the addition of the current field will make the new summation, $c^{\sum_I}$ unbounded.  Hence on the highest priority field no additional restrictions will be imposed beyond ensuring that the field itself is TVB and the coupled limiter is simplified significantly.  Subsequent fields are increasingly restricted by the constraints of previously advected fields, however only in control volumes that have values greater than zero for more than one field.  If the initial state of a field or any of the higher priority fields already breaks the boundedness criterion then the coupled limiter is unable to guarantee boundedness of the sum.

Further details of the coupled control volume algorithm can be found in \citet{wilson_phdthesis_2009}.

\subsubsection{Control Volume diffusion}

In the case where diffusion, $\tensor{\mu}$, is not zero we need to discretise the term:
\begin{equation}
-\int_{\partial v}\widehat{\vec{n}\cdot\tensor{\mu}\cdot\grad c}
\end{equation}
in equation \ref{eq:adv_diff_cv}.  As with discontinuous Galerkin methods this is complicated by the fact that $\grad c$ is now defined on the boundary of the volume.  \fluidity\ offers three strategies for dealing with this with control volumes - element based gradients, equal order Bassi-Rebay and staggered mesh Bassi-Rebay discretisations.

\paragraph{Element Gradient} \label{sec:cvegdiff}

As previously discussed the control volume boundaries intersect the parent elements at points where the parent basis functions are continuous.  The element gradient control volume diffusion scheme uses this fact to estimate the field gradients on the control volume boundaries using the parent element basis functions.  This is possible because the nodes of the parent finite element mesh and its control volume dual mesh are co-located (see Figure \ref{fig:cornerunstruct}).  This scheme is somewhat similar to standard finite volume diffusion schemes on structured meshes described in \citet{ciarlet_handbook_2000}.

\paragraph{Bassi-Rebay} \label{sec:cvbrdiff}

The Bassi-Rebay discretisation \citep{bassi1997} method was discussed in section \ref{BassiRebay} for discontinuous Galerkin discretisations.  It introduces an auxilliary variable and equation for the gradient.  As in DG, this equation can be directly solved and implicitly reinserted into the control volume equation.  However this has the disadvantage over the element gradient scheme that the sparsity structure is extended resulting in a larger matrix and more computationally expensive solves.  It may also produce spurious modes in the solution.

Both disadvantages of the simplest Bassi-Rebay discretisation may be overcome for fields, $c$, represented on piecewise linear parent finite element meshes by selecting a piecewise constant (element centred) representation of the diffusivity.  This results in the auxilliary equation for the gradient being solved for on the elements rather than at the nodes.  This can still be implicitly inserted into the equation but results in the same first order sparsity structure as the element gradient scheme.  Additionally the use of staggered finite/control volume meshes stabilises the equation and helps eliminate spurious modes.

\section{The time loop}
\label{sect:ND_time_loop}

\fluidity\ solves a coupled system of nonlinear equations with (typically)
time-varying solutions. The time-marching algorithm employed uses a
non-linear iteration scheme known as Picard iteration in which each equation
is solved using the best available solution for the other variables. This
process is then repeated either a fixed number of times or until convergence
is achieved. Figure \ref{fig:timestep}\ shows the sequence of steps in the
Picard iteration loop.

\subsection{Time notation}

It is assumed that we know the state of all variables at the $n$th timestep
and that we wish to calculate their value at the $(n+1)$st step. To take as an
example the general tracer denoted $\dvec{c}$, the value at the $n$th
timestep will be denoted $\cold$ and that at the new timestep
$\dvec{c}^{n+1}$. The Picard iteration process results in a series of
tentative results for $\dvec{c}$ at the next timestep on the basis of the
best available data. This tentative result will be written $\cnew$. At the
end of the final Picard iteration, the tentative results become final (\ie\
$\dvec{c}^{n+1}:=\cnew$). Conversely, at the start of the timestep, the only
available value of $c$ is $\cnew$ so at the start of the timestep,
$\cnew:=\cold$. This notation naturally applies \emph{mutatis mutandis}\ to
all other variables.

\begin{figure}
  \centering
  \onlypdf{\begin{pdfdisplay}}
  \begin{psmatrix}
    \psframebox{for each tracer solve the advection-diffusion equation for $\cnew$}\\
    \psframebox{evaluate the equation of state for $\rhorelax$}\\
    \psframebox{solve the momentum equation for $\ustar$}\\
    \psframebox{calculate a pressure correction $\Delta p$}\\
    \psframebox{using $\Delta p$ correct $\ustar$ to $\unew$ and update
      $\pnew$}\\
    \psframebox{perform next Picard iteration or finish timestep}
    \psset{arrows=->}
    \ncline{1,1}{2,1}
    \ncline{2,1}{3,1}
    \ncline{3,1}{4,1}
    \ncline{4,1}{5,1}
    \ncline{5,1}{6,1}
    \ncbar[angleA=0,angleB=0]{6,1}{1,1}
  \end{psmatrix}
  \onlypdf{\end{pdfdisplay}}
  \caption{Outline of the principal steps in the nonlinear iteration sequence.}
  \label{fig:timestep}
\end{figure}

\subsection{Nonlinear relaxation}\label{sect:relax}

Where a variable is used in the solution of another variable, there are two
available values of the first variable which might be employed: that at time
$n$ and the latest result for time $n+1$. For instance the velocity,
$\vec u$, is used in solving the advection-diffusion equation for
$\dvec{c}$ so $\vec{u}^n$ and $\vec{\tilde u}^{n+1}$ are available values
of $\vec u$. The choice between these values is made using the
nonlinear relaxation parameter $\thetanl$ which must lie in the interval
$[0,1]$. This allows us to define:
\begin{equation}
  \label{eq:thetanl}
  \urelax=\thetanl\vec{\tilde u}^{n+1} + (1-\thetanl)\vec{u}^n.
\end{equation}
Note at the first nonlinear iteration we generally have $\vec{\tilde u}^{n+1} = \vec{u}^n$.

\subsection{The $\theta$ scheme}
\index{theta-scheme@$\theta$-scheme}
\index{time!theta-scheme@$\theta$-scheme}
\label{sect:ND_time_theta_scheme}

The $\theta$ timestepping scheme requires the expression of a linear
combination of the known field values at the present timestep and the as-yet
unknown values at the next timestep. Taking the example of $\dvec{c}$, we write:
\begin{equation}
  \label{eq:thetac}
  \ctheta=\thetac\cnew+(1-\thetac)\cold.
\end{equation}
$\thetac$ must lie in $[0,1]$. The subscript $c$ in $\thetac$ indicates that it is 
generally possible to choose different theta timestepping schemes for different solution
variables.

\section{Time discretisation of the advection-diffusion equation}
\label{sect:ND_time_disct_adv_diff}

Regardless of the spatial discretisation options used, the
advection-diffusion equation produces a semi-discrete matrix equation of the
following form:

\begin{equation}
  \mat{M} \ddt{\dvec{c}}+\mat{A}(\vec{u})\dvec{c}+\mat{K}\dvec{c}=\dvec{r},
\end{equation}
in which $M$ is the mass matrix, $\mat{A}(\vec{u})$ is the advection
operator, $\mat{K}$ is the diffusion operator and $\dvec{r}$ is the
right-hand side vector containing boundary, source and absorption terms. For continuous
Galerkin, the matrices take the following form:
\begin{equation}
  \mat{M}_{ij}=\int_\Omega \phi_i\phi_j, \quad
  \mat{A}_{ij}=-\int_\Omega \grad\phi_i\cdot\vec{u} \phi_j, \quad
  \mat{K}_{ij}=\int_\Omega \grad\phi_i\cdot \tensor{\mu}\cdot\grad \phi_j.
\end{equation}

This is discretised using a classical $\theta$ scheme while $\vec u$ is as
given in equation \eqref{eq:thetanl}:
\begin{equation}
  \mat{M} \frac{\cnew-\cold}{\Delta t}
  +\mat{A}(\urelax)\ctheta+\mat{K}\ctheta=\dvec{r}^{n+\theta_c}.
\end{equation}
Here, $\dvec{r}^{n+\theta_c}$ indicates that the boundary condition
functions will be evaluated at $\theta_c\Delta t$ after the time at timestep
$n$. Using equation \eqref{eq:thetac}, this can be rearranged as a single
matrix equation for the unknown vector $\cnew$:
\begin{equation}
  \left(M+\thetac\Delta t\left(\mat{A}(\urelax)+\mat{K}\right)\right)\cnew
  =\left(M-(1-\thetac)\Delta
    t\left(\mat{A}(\urelax)+\mat{K}\right)\right)\cold +\dvec{r}^{n+\theta_c}.
\end{equation}
Fluidity actually uses a somewhat different rearrangement of the equations
however this is an implementation detail which has no impact for the user.

\subsection{Discontinuous Galerkin}
\index{time!advection subcycling}

The slope limiters used with the discontinuous Galerkin formulation only
guarantee a bounded solution in conjunction with an explicit advection
scheme. While the whole equation could be treated explicitly, it can be advantageous
to be able to treat only certain terms explicitly while others are treated implicitly.
To achieve this, the equation is considered in two stages: first the
tracer is advected, then diffusion occurs. This produces the following
system:
\begin{gather}
  \mat{M} \frac{\cstar-\cold}{\Delta t}
  +\mat{A}(\urelax)\cold=\dvec{r}_D^{n+\theta_c}\label{eq:adv_explicit}\\
  \mat{M} \frac{\cnew-\cstar}{\Delta t}
  +\mat{K}\ctheta=\dvec{r}_N^{n+\theta_c}+\dvec{r}_s^{n+\theta_c}\label{eq:no_adv},
\end{gather}
where now $\dvec{r}$ has been split into Dirichlet and Neumann boundary
components, and a source component. Equation \eqref{eq:no_adv}\ can be
solved directly in exactly the manner of the preceding section however the
explicit Euler scheme shown in equation \eqref{eq:adv_explicit}\ is subject
to a tight CFL criterion. Accordingly, the timestep is split into $n$
subtimesteps to satisfy a user-specified Courant number and the following
equation is solved for each:
\begin{equation}
  \mat{M} \dvec{c}_{\mathrm{new}}=\left(\mat{M}-\frac{\Delta t}{n}\mat{A}(\urelax)\right) \dvec{c}_{\mathrm{old}}+\dvec{r}_D^{n+\theta_c}.
\end{equation}
At the start of the timestep, $\dvec{c}_{\mathrm{old}}:=\cold$ and at the
end of the timestep $\cstar:=\dvec{c}_{\mathrm{new}}$.
Since the discontinuous Galerkin mass matrix is block diagonal with only the
nodes on each element being connected, it is trivial to construct
$\mat{M}^{-1}$ so this equation may be solved trivially with minimal cost. Note also that the
matrix $\mat{M}-(\Delta t/n)\mat{A}(\urelax)$ is constant within one timestep
so assembling and solving each subtimestep reduces to two matrix
multiplies and a vector addition.

If a slope limiter is employed, the slope limiter is applied to
$\dvec{c}_{\mathrm{new}}$ after each subtimestep.

\subsection{Control Volumes} \label{sec:cvtemp}

Advection subcycling based upon a CFL criterion or a fixed number of subcycles is also available for control volume discretisations, although in this case the $\theta$ value is applied globally so no advection diffusion splitting takes place.  This is generally applied to explicit discretisations of the advection equation.

When face value limiting (see section \ref{sec:cvlimiting}) is used in more implicit control volume discretisations a non-linearity is introduced by the requirement to `test' a face value, $c_{f_k}$, against an upwind value value, $c_{u_k}$, which must be estimated.  This restricts any such high order or limited face values to the right hand side of the equation, which severely limits the timestep that can be used.  To overcome this a lower order implicit pivot face value is introduced and the face value in equation \ref{eq:adv_diff_cv} is replaced by:
\begin{equation}
c_{f_k} = \theta_p c_{f_k}^{LO n+1} + \theta \tilde{c}_{f_k}^{HO} + \theta c_{f_k}^{HO n} - \theta_p \tilde{c}_{f_k}^{LO}
\end{equation}
where $c_{f_k}^{LO}$ is the low order face value and $\tilde{c}_{f_k}^{HO}$ and $\tilde{c}_{f_k}^{LO}$ are the current best estimates, based on the most recent solution, of the high order and low order face values respectively \citep[see ][ for further details]{leveque_finite-volume_2002}.

In \fluidity\ the low order pivot value uses first order upwinding (see section \ref{sec:cv_fou}) and the pivot implicitness factor, $\theta_p$, defaults to $1$.  This implicit pivot is then used to overcome the timestep restriction and an extra advection iteration loop is introduced to update the values of $\tilde{c}_{f_k}^{HO}$ and $\tilde{c}_{f_k}^{LO}$.  If this converges, then after a number of iterations $c_{f_k}^{LO n+1} \approx \tilde{c}_{f_k}^{LO}$ and $c_{f_k} \approx \theta c_{f_k}^{HO n+1} + \theta c_{f_k}^{HO n}$.  Hence an implicit high order face value is achieved.  If the iteration does not converge, either due to too few advection iterations being selected or as a result of non-convergent behaviour in the face value limiter, then $c_{f_k}$ is just a linear combination of low order and high order face values.  This still results in a valid face value, although as is generally the case for implicit advection methods, it is not possible to guarantee the boundedness of this scheme.

\section{Momentum Equation}
\index{momentum equation!discretised}

\label{Sect:ND_momentum_equation}
The discretisation of the momentum equation in non-conservative
form \eqref{nonconmom} is very similar to that of the advection-diffusion 
equation. Assuming a tensor form for viscosity, we
can write it in the same matrix form as
\eqref{eq:cg_adv_diff_mat}
\begin{equation}\label{eq:momentum_equation}
  \mat{M} \ddt{\vec u}
    +\mat{A}(\vec{u})\dvec{u}+\mat{K}\dvec{ u}
    +\matC \dvec p
    =0,
\end{equation}
with a mass matrix $\mat{M}$, advection matrix $\mat{A}$,
viscosity matrix $\mat{K}$ and pressure gradient matrix $\matC$.
For a continuous galerkin discretisation of
velocity and ignoring boundary conditions, $\mat{M}, \mat{A}$ and $\mat{K}$
are given by:
\begin{equation}
  \mat{M}_{ij}=\int_\Omega \rho\bmphi_i\cdot\bmphi_j, \quad
  \mat{A}_{ij}=\int_\Omega \bmphi_i\cdot\left(\rho\vec{u}\cdot\grad \bmphi_j\right), \quad
  \mat{K}_{ij}=\sum_{\alpha,\beta,\gamma} \int_\Omega
    \left(\partial_\beta\bmphi_{i,\alpha}\right) \tensor{\mu}_{\beta\gamma}
      \left(\partial_\gamma \bmphi_{j,\alpha}\right),
\end{equation}
where $\alpha,\beta$ and $\gamma$ are summed over the spatial dimensions.
Dirichlet and Neuman boundary conditions are implemented by
surface integrals over the domain boundary in the same way
as for the advection-diffusion equation. The discontinuous Galerkin
discretisation is again the same as that for advection-diffusion
involving additional integrals over the faces of the elements.

The pressure gradient term is new. In \fluidity\ it is possible
to use a different set of test/trial functions for pressure
and velocity --- a so-called mixed element discretisation. From this section
on, we will use the notation $\bmphi_i$ for the velocity
basis functions and $\psi_i$ for the pressure basis functions.
%% See section \ref{later} for a discussion of suitable element-pairs.

The pressure gradient matrix is then simply given by
\begin{equation}
  \matC_{ij}=\int_\Omega \bmphi_i\cdot\grad\psi_j \label{pgmatrix}
\end{equation}
In fact this is a vector of matrices, where each matrix corresponds
to one of the derivatives contained in the $\grad$ operator. Note that \fluidity\ only supports
a continuous pressure discretisation so that the same
pressure gradient matrix is well defined for a discontinuous Galerkin
velocity.

\subsection{Boussinesq approximation}
\index{Boussinesq!approximation!discretised}

The discretisation of the velocity equation \eqref{mtm}
in the Boussinesq approximation is given by simply
dropping the density in the mass and advection terms:
\begin{equation*}
  \mat{M} \ddt{\dvec u}
    +\mat{A}(\vec{u})\dvec{u}
    +\mat{Cor}~\dvec u
    -\mat{K}\dvec{u}
    +\matC \dvec p
    =\dvec{b}(\rho')+\dvec{\vec F},
\end{equation*}
where (again assuming CG and ignoring boundary conditions):
\begin{equation*}
\begin{split}
  \mat{M}_{ij}=\int_\Omega \bmphi_i\cdot\bmphi_j, \quad
  \mat{A}_{ij}=\int_\Omega \bmphi_i\cdot\left(\bmu\cdot\grad \bmphi_j\right), \quad
  \mat{Cor}_{ij}=
    \int_\Omega \bmphi_i\cdot\left(2\vec\Omega\times\bmphi_j\right), \\
  \mat{K}_{ij}=\sum_{\alpha,\beta,\gamma} \int_\Omega
    \left(\partial_\beta\bmphi_{i,\alpha}\right) \tensor{\nu}_{\beta\gamma}
      \left(\partial_\gamma \bmphi_{j,\alpha}\right),
  \matC_{ij}=\int_\Omega \bmphi_i\cdot\grad\psi_j, \quad
    \dvec{b}_i(\rho')=\int_\Omega \bmphi_i\cdot\vec g\rho',\quad
    \dvec{F}_i=\int_\Omega \bmphi_i \cdot\vec F
\end{split}
\end{equation*}

\section{Pressure equation for incompressible flow}
\index{pressue!discretisation}
\label{Sect:ND_pressure_equation}

For incompressible flow the momentum equation needs to be solved
in conjunction with the continuity equation
$\div\vec u=0$. The discretised weak form of
the continuity equation, integrated by parts, is given by
\begin{equation*}
  \sum_j \int_\Omega \left(\grad\psi_i\right)\cdot\bmphi_j u_j -
    \int_\dOmega \psi_i\bmphi_j\cdot\vec n u_j = 0.
  \label{continuity_byparts}
\end{equation*}
Similar to the advection equation, Dirichlet boundary conditions for the
normal component of the velocity can be implemented by replacing
the $\vec u\cdot\vec n$ in the boundary integral by its prescribed
value $g_D$. For this purpose we split up the
boundary $\dOmega$ into a part
$\dOmega_D$ where we prescribe $\vec u\cdot\vec n=g_D$ and
$\dOmega_{\mathrm{open}}$ where the normal component is left free.

Note that the test functions $\psi_i$ we use here are the
same as the pressure trial functions used above. As a consequence
the first term involves a matrix that is the
transpose of \eqref{pgmatrix}. Therefore if we redefine the pressure
gradient matrix as
\begin{equation*}
  \matC_{ij}=\int_\Omega \bmphi_i\cdot\grad\psi_j
    -\int_{\dOmega_{\mathrm{open}}} \bmphi_i\cdot\vec n\psi_j ,
\end{equation*}
we can write the continuity equation in the following matrix form:
\begin{equation}
  \matC^T \dvec u=
    \vec{\mat{M}}_D ~\dvec{g}{}_D,
    \quad\text{ with }\;
    \vec{\mat{M}}_{D,ij}=\int_{\dOmega_D} \psi_i\bmphi_j\cdot\vec n.
  \label{discrete_incompressible_continuity}
\end{equation}

The extra surface integral over $\dOmega_{\mathrm{open}}$ in
$\matC$ in the momentum equation will enforce
a $p=0$ boundary condition at this part of the boundary (together with the
boundary condition of the viscosity term this will form a no normal stress
condition). An inhomogoneous pressure boundary condition can also
be applied by adding the corresponding
surface integral term into the right-hand side of the momentum equation.
Using the transpose of the pressure gradient operator, including its
boundary terms, for the continuity equation thus automatically
enforces the correct physical boundary conditions in the normal
direction.
The boundary
conditions in the tangential direction (slip/no-slip) are independent of this choice.

\subsection{Pressure correction}\label{Sect:pressure_correction}
\index{pressure!correction}

After solving the momentum equation \eqref{eq:momentum_equation} for the
velocity using a pressure guess, the solution does not satisfy the
continuity equation. A common way to enforce the continuity is to project
the velocity to the set of divergence-free functions. After this projection
step the continuity equation is satisfied, but in general the solution does
not satisfy the momentum equation anymore, which is why the combination of
momentum solve and pressure correction has to be repeated until a
convergence criterium has been reached.
More information about pressure correction methods can be
found in \cite{gresho1988}.

The derivation of the pressure-correction step starts with two
variations of the time-discretisation of the
momentum equation \eqref{eq:momentum_equation}. The first
one is used to solve for a preliminary $\dvec{u}_*^{n+1}$:
\begin{equation}\label{eqn:presscorr_1}
\mat{M}  \frac{{\dvec{u}_*^{n+1}-\dvec{u}^{n}}}{\Delta t}
    +\mat{A}(\urelax)\dvec{u}_*^{n+\theta}+\mat{K}\dvec{u}_*^{n+\theta}
    +\matC \dvec p_*
    =0,
\end{equation}
where we use an initial guess pressure $\dvec p_*$, that may
be obtained from the previous time step (denoted by
$\dvec p^{n-\tfrac 12}$). % or an initial pressure poission solve as in section \ref{}. 
The second variation
describes the momentum equation that will be satisfied by
$\dvec{u}^{n+1}$ after the pressure correction:
\begin{equation}\label{eqn:presscorr_2}
\mat{M}  \frac{{\dvec{\tilde u}^{n+1}-\dvec{u}^{n}}}{\Delta t}
    +\mat{A}(\urelax)\dvec{u}_*^{n+\theta}+\mat{K}\dvec{u}_*^{n+\theta}
    +\matC \dvec{\tilde p}^{n+\tfrac 12}
    =0,
\end{equation}
here $\dvec{\tilde p}^{n+\tfrac 12}$ is the pressure
obtained after the pressure correction.

Note that $\mat{A}$ depends on $\vec{u}$ itself. For the definition of the
$\urelax$ term used calculate $\mat{A}$, see section \ref{sect:relax}.

Subtracting \eqref{eqn:presscorr_1} from \eqref{eqn:presscorr_2} yields:
\begin{equation}\label{eqn:presscorr_substract}
\mat{M}  \frac{{\dvec{\tilde u}^{n+1}-\dvec{u}_*^{n+1}}}{\Delta t}
    + \matC ( \dvec{\tilde p}^{n+\tfrac 12} - \dvec p_*)
    =0,
\end{equation}

Left-multipliying by $\matC^TM^{-1}$ and
some rearrangement results in:
\begin{equation*}
  \matC^T \mat{M^{-1}}\matC ( \dvec p^{n+\tfrac 12} -  \dvec p_*)
  =-\frac{\matC^T  ({\dvec{\tilde u}^{n+1}-\dvec{u}_*^{n+1}})}{\Delta t},
\end{equation*}
The left hand side of this equation contains the sought after pressure
correction $\Delta\dvec p=\dvec p^{n+\tfrac 12}-\dvec p_*$.
Taking into account the discretised incompressible continuity
\eqref{discrete_incompressible_continuity} evaluated at
$t^{n+1}$, we finally arrive at the pressure correction equation:
\begin{equation}
 \matC^T \mat{M^{-1}}\matC~\Delta\dvec p
   =\frac{ \matC^T \dvec{u}_*^{n+1}-\vec{\mat{M}}_D ~\dvec{g}{}_D}{\Delta t},
   \label{eqn:pressure_correction}
\end{equation}
This discrete Poisson equation can now be solved
for $\Delta\dvec p$, after which the pressure and velocity can
be updated by:
\begin{equation*}
\begin{split}
  \dvec{\tilde p}^{n+\tfrac 12} &= \dvec p_* + \Delta\dvec p, \\
  \dvec{\tilde u}^{n+1} &= \dvec{u}_*^{n+1}
  -\Delta t  M^{-1} \matC \Delta\dvec p
\end{split}
\end{equation*}
Note that by construction the obtained $\dvec{\tilde u}^{n+1}$ and
$\dvec{\tilde p}^{n+\tfrac 12}$ satisfy both the continuity equation
\eqref{discrete_incompressible_continuity} and the momentum equation
\eqref{eqn:presscorr_2}. This momentum equation however still used
the intermediate $\dvec u_*$ and $\dvec p_*$.
A more accurate answer can therefore
be obtained using multiple non-linear iterations within a
time-step. At the beginning of the subsequent non-linear iterations
the pressure $\dvec p_*$ and the advective velocity
$\dvec u_*^{n+\thetanl}$ are updated using the best available values
$\dvec{\tilde p}^{n+\tfrac 12}$ and $\dvec{\tilde u}^{n+1}$ from the previous
non-linear iteration.

\subsubsection{Inverting the mass matrix}
\label{Sect:ND_cg_mass_lumping}
\index{mass lumping}
The pressure correction equation \eqref{eqn:pressure_correction} contains
the inverse of the mass matrix $\mat{M}$. For a continuous Galerkin
discretisation of velocity, this inverse will result in a dense
matrix, so in general we do not want to explicitly construct
this. This can be avoided by approximating the mass matrix by the
so-called \emph{lumped} mass matrix:
\begin{equation*}
  \mat{M}_{L,ii}=\sum_k \mat{M}_{ik},\quad \mat{M}_{Lij\neq i}=0.
\end{equation*}
This lumped mass matrix $\mat{M}_L$ replaces $\mat{M}$ in the
discretised momentum equation. Since the lumped mass matrix is
diagonal it is trivial to invert. This lumping procedure is more
often used to avoid mass matrix inversions, for instance in
Galerkin projections. The lumping procedure is conservative,
but leads to a loss in accuracy.

With a discontinuous Galerkin discretisation of velocity, the mass matrix
becomes easier to invert. This is because the test functions only overlap
if they are associated with nodes in the same element. Therefore the mass
matrix takes on a block-diagonal form, with $n\times n$ blocks along
the diagonal, where $n$ is the number of nodes per element. These blocks
can be independently inverted and the inverse mass matrix still has the
same block-diagonal form. As a consequence the matrix
$\matC^T M^{-1} \matC$ is sparse and can be explicitly
constructed. Moreover with a continuous $P_{n+1}$ discretisation of pressure,
and a discontinuous $P_m, m\leq n$ discretisation of velocity, we
have the following property\citep{cotter2009}:
\begin{equation*}
  \matC^T M^{-1} \matC_{ij}=\int_\Omega \grad \psi_i\cdot\grad \psi_j.
\end{equation*}
That is, the discretised pressure equation \eqref{eqn:pressure_correction}
is the same as that would be obtained from a direct $P_{n+1}$
discretisation of the continuous pressure poisson equation.

\section{Balance pressure}
\label{Sect:balance_pressure}
\index{pressure!balance}

In a balanced pressure decomposition the pressure correction equation \eqref{eqn:presscorr_substract}
is modified to:

\begin{equation}\label{eqn:bp_projection}
\mat{M}  \frac{{\dvec{\tilde u}^{n+1}-\dvec{u}_*^{n+1}}}{\Delta t}
    + \matC ( \dvec{\tilde p}_r^{n+\tfrac 12} - \dvec p_{r,*}) + \mat{C_b} \dvec p_b,
    =0,
\end{equation}

where $\dvec p_b$ is some solution for the pressure field associated with buoyancy
and Coriolis accelerations, $\dvec p_r$ is the residual pressure enforcing
incompressibility, and $\mat{C_b}$ is a balanced pressure
gradient matrix. If a solution for $\dvec p_b$ can be found via some method that
is more accurate that the pressure projection method used to solve for the
residual pressure $\dvec p_r$, then this leads to a more accurate representation of
geostrophic and hydrostatic balance. In particular, for the \Poo\
element pair, a second-order accurate solution
for $\dvec p_b$ enables a second-order accurate solution for the Helmholtz decomposition
of the velocity increment associated with the buoyancy and Coriolis accelerations, even with the introduction of
pressure stabilisation.

The pressure correction equation is the Galerkin projection of the Helmholtz
decomposition of the divergent velocity increment computed from the discretised
momentum equation. In the continuous space, this is equivalent to the Helmholtz
decomposition of the forcing terms in the momentum equation, and takes the form:

\begin{align}
  \vec{F} = \vec{F}_* - \nabla p, \\
  \nabla \cdot \vec{F} = 0,
\end{align}

where $\vec{F}_*$ are all forcing terms in the continuous momentum equation
(including advection, buoyancy, Coriolis, and any other forcings).
Decomposing the pressure into a component $p_b$ associated with the buoyancy and
Coriolis accelerations $\vec{B}_*$, and a residual component $p_r$ associated with
all other forcing terms $\vec{F}_* - \vec{B}_*$, yields:

\begin{subequations}
  \begin{equation}\label{eqn:bp_helmholtz}
    \vec{B} = \vec{B}_* - \nabla p_b,
  \end{equation}
  \begin{equation}
    \vec{F} - \vec{B} = \vec{F}_* - \vec{B}_* - \nabla p_r,
  \end{equation}
  \begin{equation}
    \nabla \cdot \vec{B} = 0,
  \end{equation}
  \begin{equation}
    \nabla \cdot \left( \vec{F} - \vec{B} \right) = 0.
  \end{equation}
\end{subequations}

Taking the divergence of equation \eqref{eqn:bp_helmholtz} yields a Poisson equation for the diagnostic
balanced pressure component $p_b$:

\begin{equation}\label{eqn:bp_poisson}
  0 = \nabla \cdot \vec{B}_* - \nabla^2 p_b.
\end{equation}

For no-slip boundary conditions on $\partial \Omega$ bounding
$\Omega$, this equation has boundary conditions \linebreak
$\left( \vec{B}_* + \nabla p_b \right) \cdot \vec{\hat{n}} = 0$ on $\partial \Omega$. This corresponds
to no acceleration of fluid parcels in the direction normal to the boundary
by the non-divergent (and dynamically significant) component of the buoyancy
and Coriolis accelerations. Performing a continuous Galerkin discretisation of equation
\eqref{eqn:bp_poisson} subject to these boundary conditions yields:

\begin{equation}\label{eqn:bp_solve}
  \int_\Omega \nabla \xi_i \nabla \xi_j \dvec p_b = \int_\Omega \nabla \xi_i \cdot \vec{B}_*,
\end{equation}

where the $\xi_i$ are the balanced pressure elemental basis functions. Hence equation
\eqref{eqn:bp_solve} can be used to gain a solution for $p_b$, which can in turn
be used in equation \eqref{eqn:bp_projection}. Since $p_b$ is computed via
a Galerkin projection of the Poisson equation for the balanced pressure,
rather than a Galerkin projection of the Helmholtz decomposition of the buoyancy
and Coriolis accelerations, LBB stability constraints do not apply in the selection
of a space for $\dvec p_b$.

\section{Free surface}
\index{free surface!weak form}

Using the fact that the pressure is zero on the free surface and therefore $p=g\eta$, the weak form of the free surface kinematic boundary \eqref{freesurf3} condition is written:

\begin{equation}\label{weakfreesurf3}
\frac{1}{g} \int_{\Omega_{fs}} {\psi_i }\frac{\partial p}{\partial t}\vec z \cdot \vec n=\int_{\Omega_{fs}} \psi_i {n \cdot \bmu} \quad
  \forall \psi_i,
\end{equation}
and $\Omega_{fs}$ is the boundary where the free surface is to be applied. This condition is now incorporated into the boundary integrals of the continuity equation \eqref{continuity_byparts} giving:

\begin{equation}\label{continuity_byparts_withfs}
  -\int_\Omega \grad\psi_i \cdot \vec u
  + \int_{\partial\Omega_D} \psi_i u_n
  + \frac{1}{g} \int_{\partial\Omega_{FS}}  {\psi_i }\frac{\partial p}{\partial t}\vec z \cdot \vec n
  + \int_{\partial\Omega\setminus\partial\Omega_D\setminus\partial\Omega_{FS}} \psi_i\vec n\cdot \vec u
  = 0,\quad
  \forall \psi_i.
\end{equation}

Following the spatial and temporal discretisation described in section \ref{Sect:ND_pressure_equation}, 
the matrix form of the continuity equation now includes the free surface boundary term (compare with 
\eqref{discrete_incompressible_continuity}):
\begin{equation}
\theta C^T u^{n+1} + (1-\theta) C^T u^n + M_s \frac{\hat p^{n+1}-\hat p^n}{g \Delta t}=0
\end{equation}
Repeating the steps from section \ref{Sect:pressure_correction}, the pressure correction equation takes the form:
\begin{equation}
\left(\theta C^T(\frac{M_u}{\Delta t})^{-1} \theta C + \frac{M_s}{g(\Delta t)^2}\right)\Delta \hat p = -\frac{\theta C^T u_*^{n+1} + (1-\theta)C^Tu^n}{\Delta t}-\frac{M_s}{g(\Delta t)^2}(\hat p_*^{n+1}-\hat p^n)
\end{equation}

\section{Linear solvers} \label{ND_Linear_solvers}
\index{linear solvers}
The discretised equations (such as \eqref{eq:cg_adv_diff_mat} or \eqref{eq:momentum_equation}) form
a linear system of equations that can be written
in the following general form:
\begin{equation*}
  \mat{A}\dvec{x}=\dvec{b},
\end{equation*}
where $\mat{A}$ is a matrix, $\dvec{x}$ is a vector of the values
to solve for (typically the values at the nodes of a field $x$),
and $\dvec{b}$ contains all the terms that do not depend on
$x$. One important property of the matrices that come from
finite element discretisations is that they are very \emph{sparse},
\ie most of the entries are zero. For the solution of large
sparse linear systems so called iterative methods are usually employed
as they
avoid having to explicitly construct the inverse of the matrix
(or a suitable decomposition thereof),
which is generally dense and therefore costly to compute
(both in memory and computer time).

\subsection{Iterative solvers}
\index{linear solvers}
\index{solvers|see{linear solvers}}
\index{linear solvers!iterative}
Iterative methods try to solve a linear system by a sequence
of approximations $\dvec{x}^k$ such that $\dvec{x}^k$ converges to
the exact solution $\dvec{\hat x}$. In each iteration one can
calculate the \emph{residual}
\begin{equation*}
  \dvec{r}^k = \dvec{b} - \mat{A}\dvec{x}^k.
\end{equation*}
When reaching the exact solution $\dvec{x}^k\to\dvec{\hat x}$ the
residual $\dvec{r}^k\to 0$. The following important relation holds
\begin{equation}\label{eq:residual_error}
  \dvec{r}^k = \mat{A}\left( \dvec{\hat x} - \dvec{x}^k \right).
\end{equation}

\subsubsection{Stationary iterative methods} \label{sec:stationary_iterative_methods}
Suppose we have an approximation $\mat{M}$ of $\mat{A}$, for which
the inverse $\mat{M}^{-1}$ is easy to compute, and we apply this inverse
to \eqref{eq:residual_error}, we get the following approximation of the
error in each iteration
\begin{equation}
  \dvec{e}^k=\dvec{\hat x}-\dvec{x}^k \approx \mat{M}^{-1}\dvec{r}^k
    \label{eq:iterative_solver_error_estimate}
\end{equation}
This leads to an iterative method of the following form
\begin{equation}\label{eq:stationary_iterative_methods}
  \dvec{x}^{k+1}=\dvec{x}^{k} + \mat{M}^{-1} \dvec{r}^{k}.
\end{equation}
Because the same operator is applied at each iteration, these are
called \emph{stationary methods}.

\index{Jacobi iteration}
A well known example is the
\emph{Jacobi} iteration, where for each row $i$ the associated unknown $x_i$
is solved approximately by using the values of the previous iteration
for all other $x_j, j\neq i$. So in iteration $k$ we solve for $x^{k+1}_i$ in
\begin{equation*}
  \sum_{j<i} \mat{A}_{ij} x^k_j + \mat{A}_{ii} x^{k+1}_i +
  \sum_{j>i} \mat{A}_{ij} x^k_j = b_i
\end{equation*}
Using the symbols $\mat{L}, \mat{U}$ and $\mat{D}$ for respectively
the lower and upper diagonal part, and the diagonal matrix,
this is written as
\begin{equation*}
  \mat{L} \dvec{x}^k +
  \mat{D} \dvec{x}^{k+1} +
  \mat{U} \dvec{x}^k = \dvec b.
\end{equation*}
Because when solving for $x^{k+1}_i$ all the $x^{k+1}_{j<i}$ are already
known, one could also include these in the approximate solve. This leads
to the \emph{Gauss-Seidel} iterative method
\index{Gauss-Seidel iteration}
\begin{equation*}
  \mat{L} \dvec{x}^{k+1} +
  \mat{D} \dvec{x}^{k+1} +
  \mat{U} \dvec{x}^k = \dvec b.
\end{equation*}
After some rewriting both can be written in the form of
\eqref{eq:stationary_iterative_methods}
\begin{equation}
\begin{split}
  \text{Jacobi:   } & \dvec{x}^{k+1}=\dvec{x}^k + \mat{D}^{-1} r^k, \\
  \text{Gauss-Seidel forward:  } & \dvec{x}^{k+1}=\dvec{x}^k + \left(\mat{L}+\mat{D}\right)^{-1} r^k, \\
  \text{Gauss-Seidel backward: } & \dvec{x}^{k+1}=\dvec{x}^k + \left(\mat{U}+\mat{D}\right)^{-1} r^k.
\end{split}
\end{equation}

\subsubsection{Krylov subspace methods} \label{sec:krylov_subspace_methods}
\index{Krylov subspace methods}
\index{linear solvers!Krylov subspace methods}
Another class of iterative methods are the so called
\emph{Krylov Subspace} methods. Consider the following very
simple iterative method
\begin{equation*}
  \dvec{x}^{k+1}=\dvec{x}^k + \alpha_k \dvec{r}^k,
\end{equation*}
where $\alpha_k$ is a scalar coefficient which,
in contrast to that used in stationary methods, may be different in each
iteration. It is then easy to show that
\begin{equation*}
  \dvec{r}^{k+1}=\dvec{r}^k+\alpha_k \mat{A}\dvec{r}^k.
\end{equation*}
Since therefore the residual in each iteration is the linear
combination of the residual in the previous iteration and $\mat{A}$
applied to the previous residual, one can further derive that the
residual is a linear combination of the following vectors:
\begin{equation}\label{krylov_vectors}
  r^0, \mat{A}\dvec{r}^{0}, \mat{A}^2\dvec{r}^0, \ldots, \mat{A}^k\dvec{r}^0.
\end{equation}
The subspace spanned by these vectors is called the \emph{Krylov subspace}
and \emph{Krylov subspace} methods solve the linear system by choosing
the optimal set of coefficients $\alpha_k$ that minimises the residual.

\index{GMRES}
A well known, generally applicable Krylov method is GMRES
(Generalised Minimum RESidual, see \citet{saad1993}). Because the approximate solution
is built from all vectors in \eqref{krylov_vectors}, all of these need
to be stored in memory. For solves that require a large number of
iterations this will become too expensive. Restarted GMRES therefore sets
a maximum of those vectors to be stored (specified by the user),
after reaching this maximum the corresponding coefficients are fixed
and a new Krylov subspace is built. This typically leads to a
temporary decay in the convergence

\index{conjugate gradient method}
\index{symmetric positive definite (SPD)}
A very efficient Krylov method that only works for symmetric
positive definite (SPD) matrices is the Conjugate
Gradient (CG) method (see \citet{shewchuk1994} for
an excellent non-expert explanation of the
method). An SPD matrix is a symmetric matrix for which
\begin{equation*}
  \langle \dvec{x}, \mat{A}\dvec{x}\rangle \geq 0, \quad \forall\dvec{x}\in\mathbb{R}^n.
\end{equation*}
Using this property the CG method can find an optimal solution
in the Krylov subspace without having to store all of its
vectors. If the matrix is SPD, CG is usually the most
effective choice. Linear systems to be solved in \fluidity\ that are SPD
comprise the pressure matrix (only for incompressible flow), and the
diffusion equation.

\subsection{Preconditioned Krylov subspace methods} \label{ND_Preconditioners}
\index{preconditioners}
\index{linear solvers!preconditioners}
Because Krylov methods work by repeatedly applying the matrix
$A$ to the initial residual, eigenvectors with large
eigenvalues will become dominant in the subsequent iterations, whereas
eigenvectors with small eigenvalues are rapidly ``overpowered''. This
means the component of the error associated with small eigenvalues
is only very slowly reduced in a basic Krylov method. A measure for the
spread in the magnitude of eigenvalues is the so called
\emph{condition number}
\begin{equation*}
  C_{\mat{A}}=\frac{ \lambda_{\text{largest}} }
    { \lambda_{\text{smallest}} },
\end{equation*}
the ratio between the smallest and largest eigenvalue. A large
condition number therefore means the matrix system will be hard to solve.

A solution to this problem is to combine the stationary methods of
section \ref{sec:stationary_iterative_methods} with the
Krylov subspace methods of section \ref{sec:krylov_subspace_methods}

By pre-multiplying the equation
$\mat{A}\dvec{x}=\dvec{b}$ by the approximate inverse
$\mat{M}^{-1}$, we instead solve for
\begin{equation*}
  \mat{M}^{-1}\mat{A}\dvec{x} = \mat{M}^{-1}b.
\end{equation*}
If $\mat{M}^{-1}$ is a good approximation of the inverse of $\mat{A}$,
then
$\mat{M}^{-1}\mat{A}\approx I$ and therefore $\mat{M}^{-1}\mat{A}$
should have a much better condition number than the
original matrix $A$. This way of transforming the equation to improve
the conditioning of the system is referred to as preconditioning. Note
that we in general do not compute $\mat{M}^{-1}\mat{A}$ explicitly
as a matrix, but instead each iteration
apply matrix $A$ followed by a multiplication
with $\mat{M}^{-1}$, the preconditioner.

For SPD matrix systems solved with CG we can use a change
of variables to keep the transformed system SPD as long as the
preconditioner $\mat{M}^{-1}$ is SPD as well.

\subsubsection{Multigrid methods}
\index{multigrid}
\index{multigrid methods}
\index{algebraic multigrid (AMG)}
Even though simple preconditioners such as SOR
will improve the conditioning of the system
it can be shown that they don't work very well on
systems in which multiple length scales are
present. The preconditioned iterative method will rapidly
reduce local variations in the error, but the larger
scale, smooth error only decreases very slowly. Multigrid methods
(see e.g. \cite{trottenberg2001} for an introduction)
tackle this problem by solving the system on a hierarchy of
fine to coarse meshes.

Because \fluidity\ is based on a fully
unstructured mesh discretisation only so called algebraic multigrid (AMG)
methods are applicable (see e.g. \citet{stueben2001} for an
introduction). An AMG method that in general gives very good
results for most \fluidity\ runs is the smoothed aggregation approach by
\citet{vanek1996} (available in \fluidity\ as the ``mg''
preconditioner). For large scale ocean simulations a specific multigrid
technique, called vertical lumping\citep{kramer2010},
has been developed, that deals with the large separation between
horizontal (barotropic) and vertical modes in the pressure equation.

\subsection{Convergence criteria}
\index{linear solvers!convergence criteria}
\index{convergence criteria}
In each iterative method we need some way of telling when to stop.
As we have seen in equation \eqref{eq:iterative_solver_error_estimate}
the preconditioner applied to the residual gives a good estimate of the
error we have. So a good stop condition might be
\begin{equation*}
  \|\mat{M}^{-1} \dvec{r}^k\| \leq \epsilon_{\text{atol}},
\end{equation*}
with a user specified
\emph{absolute tolerance} $\epsilon_{\text{atol}}$, a small value
that indicates the error we are willing to tolerate.

One problem is that we quite often don't know how big the
typical value of our field is going to be (also it might change in time),
so we don't know how small $\epsilon_{\text{atol}}$ should be.
A better choice is therefore to use a \emph{relative tolerance}
$\epsilon_{\text{rtol}}$
which relates the tolerated error to a rough order estimate of the answer:
\begin{equation}\label{eq:convergence_rel_tol}
  \|\mat{M}^{-1} \dvec{r}^k\| \leq
    \epsilon_{\text{rtol}}~ \|\mat{M}^{-1} \dvec{b}\|.
\end{equation}

In some exceptional cases the right-handside of the equation may
become zero. For instance the lock exchange problem
(see \ref{sect:lock_exchange}) starts with an unstable equilibrium in which the righthand
side of the momentum equation will be zero. The right solution in this case
is of course $\dvec{x}=0$, but due to numerical round off the solver may
never exactly reach this,
and therefore never satisfy \ref{eq:convergence_rel_tol}. In this case
it is best to specify both a relative and an absolute tolerance.

Note, that the stopping criterion is always based on an \emph{approximation}
of the actual error. Especially in ill-conditioned systems this may not
always be a very good approximation. The quality of the error estimate is
then very much dependent on the quality of the preconditioner.

\section{Algorithm for detectors (Lagrangian trajectories)}
\index{Lagrangian trajectories} \index{detectors!Lagrangian}
Detectors can be set in the code at specific positions where the user wants
to know the values of certain variables at each time step. They are virtual
probes in the simulation and can be fixed in space (static detectors) or can
move with the flow (Lagrangian detectors). The configuration of detectors in
\fluidity\ is detailed in chapter
\ref{chap:configuration}. This section summarises the method
employed in the calculation of the new position of each Lagrangian detector
as it moves with the flow.

The second-order Runge-Kutta algorithm indicated in \eqref{eq:detectors_RK}
is used to advect the Lagrangian detectors with the flow. This is similar to
that used by \cite{walters2007}.

% \begin{equation}
% \mathsf{x^{n+1}} =  \mathsf{x^{n}} + \Delta t \mathsf{u}(\mathsf{x^{n}} + \frac{\Delta t }{2} \mathsf{u}(\mathsf{x^{n}}))
% \label{eq:detectors_RK}
% \end{equation}

\begin{equation}\label{eq:detectors_RK}
\bmx^{n+1} =  \bmx^{n} + \Delta t \bmu\left(\bmx^{n} + \frac{\Delta t }{2} \bmu(\bmx^{n})\right),
\end{equation}
where $\bmx^{n}$ is the initial position of the detector at the beginning of the time step, $\bmx^{n+1}$ the final position of the detector at the end of the time step, $\Delta t$ is the model or fundamental time step and $\bmu(\bmx^{n})$ is the average of the velocity at position $\bmx^{n}$ at the beginning and at the end of $\Delta t$.

In combination with the previous algorithm, a bisection method is employed consisting of bisecting the fundamental time step multiple times as the detector is moved from element to element in the mesh until all the bisected or sub-time steps add up to the fundamental time step. Figure \ref{fig:bisection_menthod} illustrates the bisection method. The procedure consists of finding out the new position of the detector after $\Delta t$ and if it is outside the initial element, reduce $\Delta t$ and find the exit point from the element and the time required to traverse the element. Then repeat the procedure for the next element until all the sub time steps add up to the model time step. The exit point from an element is found via an iterative process stopping when the detector is at the edge between two elements within a small tolerance.
%
\begin{figure}[ht]
  \centering
  \xfig{numerical_discretisation_images/bisection_method}
  %\fig[width=.7\textwidth]{numerical_discretisation_images/bisection_method}
  \caption{A sketch representing the bisection method used in combination with a second order Runge-Kutta algorithm to advect the Lagrangian detectors with the flow.}
  \label{fig:bisection_menthod}
\end{figure}


