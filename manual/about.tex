\chapter{About this manual}

\section{Introduction}

This document attempts to give an introduction to the use of the
\fluidity/ICOM code for CFD and ocean modelling applications.  Chapter \ref{chap:gettingstarted}
gives introductory material and a step by step guide on how to run and
visualise a simulation.  Chapter \ref{chap:model_physics} describes the
mathematical formulation of the model including the various equations which
may be solved with \fluidity. The following chapter, chapter
\ref{chap:numerical_discretisation} describes the numerical algorithms
employed in the model. Chapter
\ref{chap:configuration} describes the configuration options which are
available in \fluidity\ and serves as the primary reference for users setting
up simulations. Chapter \ref{chap:Adaptivity} documents the adaptive
algorithms which are available in \fluidity.

Although this document may of course be printed, viewing it on screen may be
wise as it allows colour images to be viewed, and links between sections and
parameters should work. Also, where possible figures are `vectorised' so
that if viewed electronically it is possible to zoom right in to see the
structure of meshes for example.

This manual is very much a work in progress. Therefore spelling, grammar,
and accuracy can not be guaranteed. Users with commit access to the \fluidity\
source tree are encouraged to make contributions directly. Other users are
invited to email comments, corrections and contributions to
\verb|m.d.piggott@imperial.ac.uk|

A further source of material may be found at \url{http://amcg.ese.ic.ac.uk/}
This points to a collection of wiki web pages that all users are able to
update and add to. 

\section{Audience and Scope}

The manual is primarily designed to enable \fluidity\ users to run \fluidity\
effectively. As such, this is the appropriate place for documentation
concerning the available configuration options of the model and the correct
method of employing them. It is also the correct place to document the
mathematical formulation of the model and the equations which it
solves. Other matters which should be covered include input and output
formats, checkpointing and visualisation.

This is not generally the appropriate forum for low-level documentation
directed at model developers. Information concerning the finite element
method and its implementation in \fluidity\ should be placed in the Femtools
manual while other implementation details could be placed on the AMCG wiki.

\section{Style guide}

\subsection{Images}

The manual is designed to be compilable to both PDF and html. This creates
particular challenges when incorporating images. One approach which is
particularly appropriate for diagrams and other images annotated with text
or equations is to generate or annotate the figures using xfig. Xfig files
may be automatically converted to Postscript for the PDF document and png
for the html version. If the ``special'' attribute is set on text in the
xfig document, then that text will be rendered in \LaTeX. This in particular
enables equations to be included in figures in a manner consistent with the
equations in the text.

\subsubsection{Including xfig images}

This manual defines the command \lstinline[language=TeX]+\xfig{basename}+
which will import \verb+basename.pdftex_t+ for pdf output and
\verb+basename.png+ for web output. Authors should ensure that their xfig
file has the name \verb+basename.fig+ and that \verb+basename+ is added to
the \verb+XFIG_IMAGES+ variable in the \verb+Makefile+. This will cause the
commands \verb+make fluidity_manual.pdf+ and
\verb+make fluidity_manual.html+ to also generate the pdf and png versions
of the figure respectively.

It will be observed that the \verb+\xfig+ command does not take any arguments
for figure size. This is a deliberate decision designed to ensure that the
font size matches between the figure and the text. Figures should instead be
appropriately sized in xfig.

\subsubsection{Including other figures}

For other figures, the command
\lstinline[language=TeX]+\fig[options]{basename}+ is provided. In this case,
it is the author's responsibility to provide both \verb+basename.pdf+ and
\verb+basename.png+ files. Please add \verb+basename+ to the \verb+IMAGES+
variable in the \verb+Makefile+. This will cause the image files to become
dependencies of the compiled manual.

The \lstinline[language=TeX]+options+ provided to
\lstinline[language=TeX]+\fig+ are passed straight to
\lstinline[language=TeX]+\includegraphics+ and may therefore include any
options which are legal in that context including resizing options.

\subsection{flml options}

The \verb+\option+ command is provided to format \fluidity\ option
paths. Options should be formatted according to normal Spud conventions
however it will frequently be desirable to show partial option paths not
starting from the root. In this circumstance, an ellipsis should be used to
show an unknown path component. For example, the mesh element of some
prescribed field would be \option{\ldots/prescribed/mesh} which can be input
in \LaTeX\ as \lstinline[language=TeX]+\option{\ldots/prescribed/mesh}+

\subsection{Generating pdf and html output}

The manual may be compiled to both pdf and html. For the former, type:
\begin{lstlisting}[language=bash]
  make fluidity_manual.pdf
\end{lstlisting}
and for the latter type:
\begin{lstlisting}[language=bash]
  make fluidity_manual.html
\end{lstlisting}
It may sometimes be necessary to introduce content which should only be
rendered in one or other output format. For example, long option paths
frequently defeat \LaTeX's line breaking algorithm so it may be necessary to
force line breaks in the pdf document. Since the browser is responsible for
line breaks in html, it would be inappropriate to force a linebreak in the
html output. For this purpose, the manual provides the commands
\lstinline[language=TeX]+\ifhtml{content for html}{content for pdf}+ as well
as the commands \lstinline[language=TeX]+\onlyhtml+ and
\lstinline[language=TeX]+\onlypdf+. The latter two commands take a single
argument which is only rendered for the applicable output.

\subsection{Representing source code}

Source code and commands entered in the shell can be typeset using the
\lstinline[language=TeX]+lstlisting+ environment. The environment typesets
its argument literally so unlike normal \LaTeX, spaces and carriage returns
are replicated in the output. The environment takes optional configuration
parameters of which the most important is 
\lstinline[language=TeX]+language+ which is used to select the programming
language. \LaTeX will highlight the syntax of the chosen language. Inline
commands can be typeset using\linebreak
\lstinline[language=TeX]*\lstinline[language=TeX]+command+* substituting any
applicable language. 

\subsubsection{Shell commands}

For shell commands, \lstinline[language=TeX]+language+ should be set to
\lstinline[language=TeX]+bash+. For example:
\begin{verbatim}
\begin{lstlisting}[language=bash]
dham@popper traffic > ls
box.ele   fluidity.err-0  Makefile       traffic_1.vtu  traffic.xml
box.face  fluidity.log-0  src            traffic.flml   vaf.bin
box.node  gmon.out        traffic_0.vtu  traffic.stat   vaf.dat
\end{lstlisting}
\end{verbatim}
will be rendered:
\begin{lstlisting}[language=bash]
dham@popper traffic > ls 
box.ele   fluidity.err-0  Makefile       traffic_1.vtu  traffic.xml
box.face  fluidity.log-0  src            traffic.flml   vaf.bin
box.node  gmon.out        traffic_0.vtu  traffic.stat   vaf.dat
\end{lstlisting}

\subsubsection{Other languages}

The other languages which are currently enabled are
\lstinline[language=TeX]+TeX+, for \LaTeX, \lstinline[language=TeX]+Python+,
\lstinline[language=TeX]+Make+ and \lstinline[language=TeX]+XML+. Many other
languages are supported by the \lstinline[language=TeX]+listings+ package.

\subsection{Bibliography}

Citations from the literature should be included whenever relevant. When
formatting entries in the bibliography database,
\lstinline[language=Bash]+bibliography.bib+, the preferred key is the first
author's surname in lower case followed by the full year of publication. For
example \lstinline[language=TeX]+ham2009+.

The manual uses an author-date citation style which means that it is
important to use the correct combination of
\lstinline[language=TeX]+\cite+,  \lstinline[language=TeX]+\citep+ and
\lstinline[language=TeX]+\citet+. See the \LaTeX\ \lstinline[language=TeX]+natbib+
 package documentation for more details.
\subsection{Mathematical notation conventions}

\subsubsection{Continuous Vectors and tensors}

There are two conceptually different forms of vector and tensor which occur
in the finite element method. The first is for quantities, such as velocity,
which are vector-valued in the continuum. These should be typeset in italic
bold: $\bmu$. The \lstinline[language=TeX]+\vec+ command has been redefined
for this purpose so a vector quantity named $\vec{b}$ would be typed
\lstinline[language=TeX]+\vec{b}+. However, a large number of
frequently-used vector quantities have convenience functions pre-defined in
\lstinline[language=bash]+notation.tex+. These have the name
\lstinline[language=TeX]+\bm+\textit{n}\ where \textit{n}\ is the symbol to
be typeset. Examples include \lstinline[language=TeX]+\bmu+ ($\bmu$) and
\lstinline[language=TeX]+\bmphi+ ($\bmphi$).

Continuous tensors are represented using a double overbar:
$\tensor{\tau}$. The \lstinline[language=TeX]+\tensor+ command is provided
for this purpose. Once again, convenience functions are provided for common
tensors, this time with the form
\lstinline[language=TeX]+\+\textit{n}\lstinline[language=TeX]+tens+ for
example \lstinline[language=TeX]+\tautens+ ($\tautens$) and
\lstinline[language=TeX]+\ktens+ ($\ktens$).

\subsubsection{Discrete vectors and matrices}

Vectors composed of the value of a field at each node and matrices mapping
between discrete spaces should be typeset differently from continuous
vectors and tensors. Discrete vectors should be typeset with an underline
using the \lstinline[language=TeX]+\dvec+ command. This can be combined with
\lstinline[language=TeX]+vec+ for discretisations of vector quantities. For
example \lstinline[language=TeX]+\dvec{\bmu}+ ($\dvec{\bmu}$) for the
discretised velocity vector.

Matrices should be typeset as upright upper case letters. The
\lstinline[language=TeX]+\mat+ command is available for this purpose. For
example \lstinline[language=TeX]+\mat{M}+ produces $\mat{M}$.

\subsubsection{Derivatives}

The full derivative and the material derivative should be typeset using an
upright $\d$ and $\D$ respectively. The \lstinline[language=TeX]+\d+ and
\lstinline[language=TeX]+\D+ commands are provided for this purpose. There
are also a number of functions provided for typesetting derivatives. Each of
these functions has one compulsory and one optional argument. The
compulsory argument is the function of which the derivative is being taken,
the optional argument is the variable with respect to which the derivative
is being taken. So, for example \lstinline[language=TeX]+\ppt[q]{y}+ gives:
\begin{equation*}
  \ppt[q]{y}.
\end{equation*}
While simple \lstinline[language=TeX]+\ppt{}+ gives:
\begin{equation*}
  \ppt{}.
\end{equation*}
Table \ref{tab:derivatives}\ shows the derivative functions available.
\begin{table}[ht]
  \centering
  \begin{tabular}{lc}
    \textbf{command} & \textbf{example}\\\hline
    \lstinline[language=TeX]+\ddx[]{}+ & \rule{0pt}{4ex}$\displaystyle\ddx{}$\\
    \lstinline[language=TeX]+\ddxx[]{}+ & \rule{0pt}{4ex}$\displaystyle\ddxx{}$\\
    \lstinline[language=TeX]+\ddt[]{}+ & \rule{0pt}{4ex}$\displaystyle\ddt{}$\\
    \lstinline[language=TeX]+\ddtt[]{}+ & \rule{0pt}{4ex}$\displaystyle\ddtt{}$\\
    \lstinline[language=TeX]+\ppx[]{}+ & \rule{0pt}{4ex}$\displaystyle\ppx{}$\\
    \lstinline[language=TeX]+\ppxx[]{}+ & \rule{0pt}{4ex}$\displaystyle\ppxx{}$\\
    \lstinline[language=TeX]+\ppt[]{}+ & \rule{0pt}{4ex}$\displaystyle\ppt{}$\\
    \lstinline[language=TeX]+\pptt[]{}+ & \rule{0pt}{4ex}$\displaystyle\pptt{}$\\
    \lstinline[language=TeX]+\DDx[]{}+ & \rule{0pt}{4ex}$\displaystyle\DDx{}$\\
    \lstinline[language=TeX]+\DDxx[]{}+ & \rule{0pt}{4ex}$\displaystyle\DDxx{}$\\
    \lstinline[language=TeX]+\DDt[]{}+ & \rule{0pt}{4ex}$\displaystyle\DDt{}$\\
    \lstinline[language=TeX]+\DDtt[]{}+ & \rule{0pt}{4ex}$\displaystyle\DDtt{}$\\ 
  \end{tabular}
  \caption{Functions for correctly typesetting derivatives.}
  \label{tab:derivatives}
\end{table}

\subsubsection{Integrals}

Integrals in any number of dimensions should be typeset with an integral
sign and no measure (\ie no $\d x$ or $\d V$). The domain over which the
integral is taken should be expressed as a subscript to the integral sign
itself. The integral of $\psi$ over the whole domain will therefore be
written as:
\begin{equation*}
  \int_\Omega \phi.
\end{equation*}

\subsubsection{Units}

Units should be typeset in upright font. The \LaTeX\ package
\lstinline+units+ does this for you automagically. The correct syntax is
\lstinline[language=TeX]+\unit[value]{unit}+. For example
\lstinline[language=TeX]+\unit[5]{m}+ produces \unit[5]{m}. There are a
number of convenience functions defined for the manual to make this job
easier. These are shown in table \ref{tab:units}. Providing the value as an
argument to the unit ensures that the spacing between the value and the unit
is correct and will not break over lines. The value is an optional argument
so if there is no value, just leave it out. \lstinline+units+ does the right
thing in both text and math modes. Other convenience functions can easily be
added to \lstinline[language=bash]+notation.tex+.
\begin{table}[ht]
  \centering
  \begin{tabular}{lcc}
    \textbf{command} & \textbf{example}\\\hline
    \lstinline[language=TeX]+\m[length]+ & \m[1] \\
    \lstinline[language=TeX]+\km[length]+ & \km[1] \\
    \lstinline[language=TeX]+\s[time]+ & \s[1] \\
    \lstinline[language=TeX]+\ms[speed]+ & \ms[1] \\
    \lstinline[language=TeX]+\mss[accel]+ & \mss[1] \\
    \lstinline[language=TeX]+\K[temp]+ & \K[1] \\
    \lstinline[language=TeX]+\PSU[salin]+ & \PSU[1] \\
    \lstinline[language=TeX]+\Pa[press]+ & \Pa[1] \\
    \lstinline[language=TeX]+\kg[mass]+ & \kg[1] \\
    \lstinline[language=TeX]+\rads[ang_vel]+ & \rads[1] \\
    \lstinline[language=TeX]+\kgmm[density]+ & \kgmm[1] \\
  \end{tabular}
  \caption{Convenience functions for physical units}
  \label{tab:units}
\end{table}


\subsubsection{Abbreviations in formulae}

Abbreviations in formulae should be typeset in upright maths mode using
\lstinline[language=TeX]+\mathrm+. For example
\lstinline[language=TeX]+F_{\mathrm{wall}}+ ($F_{\mathrm{wall}}$).  
