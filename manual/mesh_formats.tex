\chapter{Mesh formats}\label{chap:mesh_formats}
\index{mesh!generation}
\index{grid|see{mesh}}

This chapter describes the information contained in a mesh file and the 
three mesh formats that can be read by \fluidity: the gmsh, triangle and 
ExodusII format.
For an overview and further pointers on how to generate these meshes
see chapter \ref{chap:meshes}.

\section{Mesh data}

A mesh describes the computational domain in which a simulation takes
place. Regardless of the mesh file format in use, the information conveyed
is essentially the same.

\subsection{Node location}

The locations of the element vertices are recorded. Usually, these have the
same dimension as the topological dimension of the mesh elements. \fluidity\
does not currently support configurations such as shells in which the node
location dimension differs from the element topology dimension.

\subsection{Element topology}

The mesh is composed of elements. In one dimension these will be intervals
with each interval joining two nodes. In two dimensions, triangles or
quadrilaterals are supported with the elements joining three or four nodes
respectively. In three dimensions, the elements can be tetrahedra or
hexahedra and will join four or eight nodes.

The element topology will store the indices of the nodes which make up
each of the elements in the mesh.

\subsection{Facets}

Facets form the surface of elements. In one dimension, the facets of an
element are its bounding nodes. In two dimensions, the facets are the edge
intervals while the facets of a three-dimensional tetrahedral element are
triangles and those of a hexahedral element are quadrilaterals. External
mesh formats tend to only supply facet topology information for facets on
the surface of each domain. For each facet specified, the node indices of
that facet will be given. These surface facets are used in combination with
surface IDs to specify the regions over which boundary conditions should be
applied.

\subsection{Surface IDs}\label{sec:surface_ids}
\index{surface ID}
Numbers can be assigned to label particular facets (boundary nodes, edges or
faces in 1, 2 or 3 dimensions respectively) in order to set
boundary conditions or other parameters. This number can then be used to
specify which surface a particular boundary condition should be applied to
in \fluidity. 

\subsection{Region IDs}\label{sec:region_ids}
\index{region ID} These are analogous to surface IDs, however they are
associated with elements rather than facets. Region IDs may be used in
\fluidity\ to specify different initial conditions or material properties to
different parts of the domain.



\section{The triangle format}\label{sec:triangle_format}
\index{mesh!file formats}

The \emph{triangle} format is a ASCII file format originally designed for 2D
triangle meshes, but it can be easily extended to different dimensions and
more complex geometries.  \fluidity\ supports a version of the triangle format
which supports 1D, 2D and 3D meshes.  The following table shows the supported
combinations of element dimension and geometry.

\begin{tabular}{ l l l }
\textbf{Dimension} & \textbf{Geometry} & \textbf{Number of vertices per element} \\ \hline
1D & Line & 2\\ 
2D & Triangles &  3 \\ 
2D & Quadrilateral & 4 \\
3D & Tetrahedra & 4 \\ 
3D & Hexahedra & 8 \\
\end{tabular}

A complete triangle mesh consists of three files: one file defining the
nodes of the mesh, one file describing the elements (for example triangles
in 2D) and one file defining the boundary parts of the mesh.

The triangle file format is very simple. Since the data is stored in ASCII,
any text editor can be used to edit the files.  Lines starting with \# will
be interpreted as a comment by \fluidity.  The filename should end with
either .node, .ele, .bound, .edge or .face.  The structure of these files
will now be explained:

\begin{description}
\item[.node file]
This file holds the coordinates of the nodes. The file structure is:

First line
\begin{lstlisting}: 
<total number of vertices> <dimension (must be 1,2 or 3)> 0 0
\end{lstlisting}
Remaining lines
\begin{lstlisting}
<vertex number> <x> [<y> [<z>]]
\end{lstlisting} 
where x, y and z are the coordinates.

Vertices must be numbered consecutively, starting from one. 

\item[.ele file] Saves the elements of the mesh. The file structure is:
\index{region ID}

First line:
\begin{lstlisting}
 <total number of elements> <nodes per element>  1
\end{lstlisting}
Remaining lines:
\begin{lstlisting} 
<element number> <node> <node> <node> ... <region id>
\end{lstlisting}  
The elements must be numbered consecutively, starting from one. Nodes are
indices into the corresponding .node file. For example in case of describing
a 2D triangle mesh, the first three nodes are the corner vertices. The
region ID can be used by \fluidity\ to set conditions on different parts of
the mesh, see section \ref{sec:region_ids}.

\item[.bound file] This file is only generated for one-dimensional meshes.
  It records the boundary points and assigns surface IDs to them. The file
  structure is:\index{surface ID}

First line:
\begin{lstlisting}
 <total number of boundary points> 1
\end{lstlisting}
Remaining lines:
\begin{lstlisting} 
<boundary point number> <node> <surface id>
\end{lstlisting}  
The boundary points must be numbered consecutively, starting from one. Nodes
are indices into the corresponding .node file. The surface ID is used by
\fluidity\ to specify parts of the surface where different boundary
conditions will be applied, see section \ref{sec:surface_ids}.

\item[.edge file] This file is only generated for two-dimensional meshes.
  It records the edges and assigns surface IDs to part of the mesh surface.
  The file structure is:

First line:
\begin{lstlisting}
 <total number of edges> 1
\end{lstlisting}
Remaining lines:
\begin{lstlisting} 
<edge number> <node> <node> ... <surface id>
\end{lstlisting}  
The edges must be numbered consecutively, starting from one. Nodes are
indices into the corresponding .node file. The surface ID is used by
\fluidity\ to specify parts of the surface where different boundary
conditions will be applied, see section \ref{sec:surface_ids}.


\item[.face file] This file is only generated for three-dimensional meshes.
  It records the faces and assigns surface IDs to part of the mesh surface. The
  file structure is:

First line:
\begin{lstlisting}
 <total number of faces> 1
\end{lstlisting}
Remaining lines:
\begin{lstlisting} 
<face number> <node> <node> <node> ... <surface id>
\end{lstlisting}  
The faces must be numbered consecutively, starting from one. Nodes are
indices into the corresponding .node file. The surface ID is used by
\fluidity\ to specify parts of the surface where different boundary
conditions will be applied, see section \ref{sec:surface_ids}.
\end{description}

To clarify the file format, a simple 1D example is shown: 
The following .node file defines 6 equidistant nodes from 0.0 to 5.0
\begin{lstlisting}
# example.node
6 1 0 0
1 0.0
2 1.0
3 2.0
4 3.0
5 4.0
6 5.0
\end{lstlisting}
The .ele file connects these nodes to 5 lines. Two regions will be defined with the IDs 1 and 2.
\begin{lstlisting}
# example.ele
5 2 1
1 1 2 1
2 2 3 1
3 3 4 1
4 4 5 2
5 5 6 2
\end{lstlisting}
Finally, the .bound file tags the very left and very right nodes as boundary
points an assigns the surface IDs 1 and 2 to them.
\begin{lstlisting}
# example.bound
2 1
1 1 1
2 6 2
\end{lstlisting}

\section{The Gmsh format}\label{sec:gmsh_format}

\fluidity\ contains  support for the Gmsh format. Gmsh is a mesh
generator freely available on the Web at (\url{http://geuz.org/gmsh/}), and 
is included in Linux distributions such as Ubuntu. 

Unlike triangle files, Gmsh meshes are contained within one file, which have
the extension \lstinline[language=bash]+.msh+. The file contents may
be either binary or ASCII.

This section briefly describes the Gmsh format, and is only intended
to serve as a short introduction. If you need further
information, please read the official Gmsh documentation
(\url{http://geuz.org/gmsh/doc/texinfo/gmsh.pdf}).
Typically Gmsh files used in \fluidity\ contain three parts: a header, a
section for nodes, and one for elements. These  are explained in more
detail below.

\subsubsection*{The header}\label{sec:gmsh_header_section}
This contains Gmsh file version information, and indicates whether 
the main data is in ASCII or binary format. This will typically look like:
\begin{lstlisting}
$MeshFormat
2.1 0 8
[Extra data here in binary mode]
$EndMeshFormat
\end{lstlisting}

From the listing above we can tell that:
\begin{itemize}
\item the Gmsh format version is 2.1
\item it is in ASCII, as indicated by the 0 (1=binary)
\item the byte size of double precision is 8
\end{itemize}
In addition, in binary mode the integer 1 is written as 4 raw bytes, to check that the endianness of the Gmsh file and the system are the same (\textit{you will rarely have to worry about this})


\subsubsection*{The nodes section}\label{sec:gmsh_nodes_section}

The next section contains node data, viz:
\begin{lstlisting}
$Nodes
number_of_nodes
[node data]
$EndNodes
\end{lstlisting}

The \lstinline+[node data]+ part contains the listing of nodes, with ID,
followed by $x$, $y$, and $z$ coordinates. This part
will be in binary when binary mode has been selected. Note that even with 2D
problems, there will be a zeroed \textit{z} coordinate.



\subsubsection*{The elements section}\label{sec:gmsh_elements_section}

The elements section contains information on both facets and regular
elements. It also varies between binary and ASCII formats. The ASCII version
is:

\begin{lstlisting}
$Elements
element1_id element_type number_of_tags tag_list node_number_list
element2_id ...
...
...
$EndElements
\end{lstlisting}
\textit{Tags} are integer properties ascribed to the element. In \fluidity,
usually we are only concerned with the first one, the physical ID. This can mean
one of two things:

\begin{itemize}
\item A region ID - in the case of elements
\item A surface ID - in the case of facets
\end{itemize}

Since Gmsh doesn't explicitly label facets or regular elements as such,
internally \fluidity\ works this out from type: eg, if there a mesh consists
of tetrahedra and triangles, then triangles must be the facets.

\subsubsection*{Internal use }
\label{sec:gmsh_internal_use_section}

\fluidity\ occasionally augments GMSH files for its own internal use. It
does this in two cases. Firstly, when generating periodic meshes 
through \lstinline+periodise+ (section \ref{sec:decomposing_meshes_periodise}), the
fourth element tag is reserved to label elements at the periodic boundary.
Secondly, when checkpointing extruded meshes, a new section called
\lstinline+$NodeData+ is created at the end of the GMSH file; this contains
essential node column ID information.
 
\section{The ExodusII format}
\label{sec:exodusii_format}
\index{exodusii}

Currently Fluidity support read-only support for serial ExodusII files. It is the default 
output mesh format of the automated mesh generation toolkit \href{http://cubit.sandia.gov/}{Cubit}.
The mesh is stored within one binary file and typically has one of the following file extensions: 
\texttt{.e}, \texttt{.E}, \texttt{.exo}, or \texttt{.EXO}.

A brief description of the structure of an ExodusII file is given. Interested readers can find a 
more detailed description of the ExodusII format in the official manual at 
\url{http://sourceforge.net/projects/exodusii/}.

\subsubsection*{The header}
\label{ssec:exodusii_header_section}

The header contains basic information about the ExodusII file, such as the number of dimensions, 
nodes, elements, blocks, sidesets and many more. Below an excerpt of the header is given.
\begin{lstlisting}
netcdf \2dbox_with_sideset {
dimensions:
    ...
    num_dim = 2 ;
    num_nodes = 23 ;
    num_elem = 35 ;
    num_el_blk = 2 ;
    ...
    num_side_sets = 4 ;
    num_side_ss1 = 5 ;
    ...
    num_el_in_blk1 = 30 ;
    num_nod_per_el1 = 3 ;
    ...
\end{lstlisting}

From this header we can identify that the file contains a 2-dimensional 
mesh (\texttt{num\_dim = 2}), with 23 nodes (\texttt{num\_nodes}), and 
35 elements (\texttt{num\_elem}). Furthermore, the mesh was divided into 
two blocks (\texttt{num\_el\_blk = 2}), whereas the first block has 
30 elements (\texttt{num\_el\_in\_blk1}) and the number of nodes of each 
element in this block is three (\texttt{num\_nod\_per\_el1 = 3}). In addition 
to the blocks, four sidesets were defined (\texttt{num\_side\_sets}), 
whereas the first sideset contains 5 elements (\texttt{num\_side\_ss1 = 5}).
The ids of the blocks and sidesets are not to be confused with the numbering 
of the blocks and sidesets above. The corresponding ids are stored in the 
\texttt{data} section of an ExodusII file, which is covered in the following 
section.

\subsubsection*{The data section}
\label{ssec:exodusii_data_section}

The data section contains the actual mesh properties, such as block and sideset ids,
the list of elements and sides of each sideset, the node connectivity of each 
defined block, and of course the element mapping of the entire mesh and the 
coordinates of the vertices. 
\begin{lstlisting}
data:
...
 ss_prop1 = 10, 11, 12, 13 ;
...
 elem_ss1 = 4, 8, 19, 20, 21 ;
...
 elem_ss2 = 2, 4, 13 ;
...
 elem_map = 1, 2, 3, 4, 5, 6, 7, 8, 9, 10, 11, 12, 13, 14, 15, 16, 17, 18, 
    19, 20, 21, 22, 23, 24, 25, 26, 27, 28, 29, 30, 31, 32, 33, 34, 35 ;
...
 eb_prop1 = 1, 2 ;
...
 connect2 =
  7, 8,
  8, 20,
  20, 23,
  23, 15,
  15, 16 ;
...
  coord =
  0.283701736685801, 0.166666666666667, 0.0265230044875502, 0.5, 
    -0.0944251982348238, -0.258225085856399, -0.5, -0.5, -0.166666666666667, 
    0.218376401545253, 0.283363015396065, 0.5, -0.166666666666667, 
    -0.25779029343402, -0.5, -0.5, 0.0259480610716054, 0.5, 0.5, -0.5, 
    -0.321383277222933, 0.166666666666667, -0.5,
  0.280428743148782, 0.5, 0.242760465386713, 0.5, -0.00227602386180678, 
    0.188837287760358, 0.5, 0.3, 0.5, -0.00202880785288154, 
    -0.284316506446615, -0.166666666666667, -0.5, -0.193411124527211, -0.3, 
    -0.5, -0.247855406465798, -0.5, 0.166666666666667, 0.1, 
    -0.00195855747202122, -0.5, -0.1 ;
\end{lstlisting}

Here \texttt{ss\_prop1} lists the id numbers of the defined sidesets, e.g.~the 
id number $10$ was assigned to the first sideset. As stated in the header, the 
first sideset contains five elements (\texttt{num\_side\_ss1 = 5}), which are 
listed in the data section under \texttt{elem\_ss1 = 4, 8, 19, 20, 21 ;}

As stated above, the mesh was divided into two blocks. Their id numbers are 1 
and 2, as can be seen under \texttt{eb\_prop1 = 1, 2;} The element connectivity 
stored for each block separately under \texttt{connect1} and \texttt{connect2}. 

Finally the coordinates of the vertices are stored under \texttt{coord = ...}

\subsection*{Use within Fluidity}
\label{ssec:exodusii_use_within_fluidity}

Once the mesh is generated, it can be read in by Fluidity by simply setting the 
mesh format in Diamond to \texttt{exodusii} and providing its file name.

The region ids and surface ids set in Diamond are the block ids and sideset ids 
in the ExodusII file respectively. 


\subsection*{Current restrictions}
\label{ssec:exodusii_restrictions}

This is work in progress and the following restrictions apply to using the ExodusII 
mesh format:
\begin{itemize}
 \item Currently only serial mesh files can be read in
 \item Currently Fluidity does not support writing ExodusII files, thus whenever
 an ExodusII mesh file is read in, the checkpointed mesh files that are written are 
 in the gmsh mesh format.
 \item FLTools such as periodise currently do not support the ExodusII mesh format.

\end{itemize}

