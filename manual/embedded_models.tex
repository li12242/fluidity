\chapter{Embedded models}\label{chap:embedded}

The parameterisations described in chapter \ref{chap:parameterisations}\ are
used to model fluids processes which cannot be resolved by the model. In
contrast, the models described in this chapter detail models embedded in
\fluidity\ which model non-fluids processes.

\section{Biology}
\label{sec:biology_model}
The Biology model in \fluidity\ contains a number of different submodels. All are
currently population level models where variables evolve under an 
advection-diffusion equation similar to that for other tracers, such as 
temperature and salinity, but modified by the
addition of a source term which contains the interactions between the
biological fields. The fluxes depend on which model is selected and which
tracers are available. 

There are two models currently distributed with \fluidity: a four-component model
and a six-component model.

\subsection{Four component model}

Figure \ref{fig:biofluxes} shows the interactions between the
biological tracers in the four component model. Nutrients (plus sunlight) are converted into
phytoplankton. Zooplankton feed on phytoplankton and detritus but in doing
so produce some nutrients and detritus so grazing also results in fluxes
from phytoplankton and detritus to nutrient and detritus. Both phytoplankton
and zooplankton die producing detritus and detritus is gradually converted to
nutrient by a remineralisation process.
\begin{figure}[hb]
  \centering
  \onlypdf{\begin{pdfdisplay}}
    \begin{psmatrix}[colsep=4,rowsep=4]
      \psframebox[framearc=.2]{\
        \begin{minipage}[c][7ex]{0.3\linewidth}
          \centering
          Phytoplankton
        \end{minipage}} &
      \psframebox[framearc=.2]{
        \begin{minipage}[c][7ex]{0.3\linewidth}
          \centering
          Zooplankton
        \end{minipage}}\\
      \psframebox[framearc=.2]{
        \begin{minipage}[c][7ex]{0.3\linewidth}
          \centering
          Nutrient
        \end{minipage}}&
      \psframebox[framearc=.2]{
        \begin{minipage}[c][7ex]{0.3\linewidth}
          \centering
          Detritus
        \end{minipage}}\\
      \ncarc[angleB=270]{->}{1,1}{1,2}
      \naput{grazing}
      \ncarc[angleA=0,angleB=180]{->}{2,1}{1,1}
      \naput{fertilisation}
      \ncarc[angleA=180,angleB=0]{->}{1,1}{2,1}
      \naput{grazing}
      \ncarc[angleA=270]{->}{2,2}{2,1}
      \naput{remineralisation}
      \ncarc[angleA=270]{<-}{2,1}{2,2}
      \naput{grazing}
      \ncarc[angleA=180,angleB=0]{->}{1,2}{2,2}
      \naput{death}
      \ncarc[angleA=0,angleB=180]{->}{2,2}{1,2}
      \naput{grazing}
      \ncarc[angleA=315,angleB=135]{<-}{2,2}{1,1}
      \naput{death}
      \ncarc[angleA=135,angleB=315]{->}{1,1}{2,2}
      \naput{grazing}
    \end{psmatrix}
  \onlypdf{\end{pdfdisplay}}
  \caption{The fluxes between the biological tracers. Grazing refers to the
    feeding activity of zooplankton on phytoplankton and detritus.}
  \label{fig:biofluxes}
\end{figure}

\subsubsection{Biological source terms}
The source terms for phytoplankton ($P$), zooplankton ($Z$), nutrient ($N$)
and detritus ($D$) respectively are given by the following expressions:
\begin{gather}
  S_P=\mathrm{R}_P -  \mathrm{G}_P -\mathrm{De}_P,\\
  S_Z=\gamma\beta(\mathrm{G}_P+\mathrm{G}_D) - \mathrm{De}_Z,\\
  S_N=-\mathrm{R}_P + \mathrm{De}_D +
  (1-\gamma)\beta(\mathrm{G}_P+\mathrm{G}_D),\\
  S_D=-\mathrm{De}_D + \mathrm{De}_P + \mathrm{De}_Z + (1-\beta)\mathrm{G}_P
  -\beta\mathrm{G}_D).
\end{gather}
The definitions of each of these terms are given below. It is significant
that the right-hand sides of these equations sum to zero. This implies that,
for a closed biological system, the total of the biological constituents is
always conserved. The terms in these equations are given in table
\ref{tab:bioparameters}. The variable terms are explained in more detail below.

\begin{table}[ht]
  \centering
  \begin{tabular}{llll}\hline
    \textbf{Symbol} & \textbf{Meaning} & \textbf{Typical value} & \textbf{Section}\\\hline
    $\mathrm{R}_P$ & phytoplankton growth rate & & \ref{sec:R_P}\\
    $\mathrm{G}_P$ & rate of zooplankton grazing on phytoplankton && \ref{sec:G_P}\\
    $\mathrm{De}_P$ & death rate of phytoplankton && \ref{sec:De_P}\\
    $\mathrm{G}_D$ & rate of zooplankton grazing on detritus && \ref{sec:G_D}\\
    $\mathrm{De}_Z$ & death rate of zooplankton && \ref{sec:De_Z}\\
    $\mathrm{De}_D$ & death rate of detritus && \ref{sec:De_D}\\
    $I$ & photosynthetically active radiation & & \ref{sec:I} \\
    $\alpha$ & sensitivity of phytoplankton to light & \unit[0.015]{m$^2$\,W$^{-1}$\,day$^{-1}$}  & \ref{sec:R_P}\\
    $\beta$ & assimilation efficiency of zooplankton & 0.75 \\
    $\gamma$ & zooplankton excretion parameter & 0.5 \\
    $\mu_P$ & phytoplankton mortality rate & \unit[0.1]{day$^{-1}$} & \ref{sec:De_P}\\
    $\mu_Z$ & zooplankton mortality rate & \unit[0.2]{day$^{-1}$} & \ref{sec:De_Z}\\
    $\mu_D$ & detritus remineralisation rate & \unit[0.05]{day$^{-1}$} & \ref{sec:De_D}\\
    $g$ & zooplankton maximum growth rate & \unit[1]{day$^{-1}$} & \ref{sec:G_P}\\
    $k_N$ & half-saturation constant for nutrient & 0.5  & \ref{sec:G_D}\\
    $k$ & zooplankton grazing parameter & 0.5 & \ref{sec:G_P}\\
    $p_P$ & zooplankton preference for phytoplankton & 0.75 & \ref{sec:G_P}\\
    $v$ & maximum phytoplankton growth rate & \unit[1.5]{day$^{-1}$} & \ref{sec:R_P}\\
    \hline
  \end{tabular}
  \caption{Meanings of symbols in the biology model. Typical values are
    provided for externally set parameters.}
  \label{tab:bioparameters}
\end{table}

\textbf{$\mathrm{R}_P$, the phytoplankton growth rate}\label{sec:R_P}

$\mathrm{R}_P$ is the growth-rate of phytoplankton which is governed by the
current phytoplankton concentration, the available nutrients and the
available light:
\begin{equation}
  \mathrm{R}_P=J\,P\,Q,
\end{equation}
where $J$ is the light-limited phytoplankton growth rate which is in turn given
by:
\begin{equation}
  J=\frac{v\alpha I}{(v^2+\alpha^2 I^2)^{1/2}}.
\end{equation}
In this expression, $v$ is the maximum phytoplankton growth rate, $\alpha$
controls the sensitivity of growth rate to light and $I$ is the available
photosynthetically active radiation.

$Q$ is the nutrient limiting factor and is given by:
\begin{equation}
  Q=\frac{N}{k_N+N},
\end{equation}
where $k_N$ is the half-saturation constant for nutrient.

\textbf{$\mathrm{G}_P$, the rate of phytoplankton grazing by zooplankton}\label{sec:G_P}
The rate at which zooplankton eat phytoplankton is given by:
\begin{equation}
  \mathrm{G}_P=\frac{g p_P P^2 Z}{k^2 + p_P P^2 + (1-p_P) D^2},
\end{equation}
in which $p_P$ is the preference of zooplankton for grazing phytoplankton
over grazing detritus, $g$ is the maximum zooplankton growth rate and $k$ is
a parameter which limits the grazing rate if the concentration of
phytoplankton and detritus becomes very low.

\textbf{$\mathrm{G}_D$, the rate of detritus grazing by zooplankton}\label{sec:G_D}
The rate at which zooplankton eat detritus is given by:
\begin{equation}
  \mathrm{G}_D=\frac{g (1-p_P) D^2 Z}{k^2 + p_P P^2 + (1-p_P) D^2},
\end{equation}
in which all of the parameters have the meaning given in the previous
section.

\textbf{$\mathrm{De}_P$, the phytoplankton death rate}\label{sec:De_P}

A proportion of the phytoplankton in the system will die in any given time
interval. The dead phytoplankton become detritus:
\begin{equation}
  \mathrm{De}_P=\frac{\mu_P P^2}{P+1},
\end{equation}
in which $\mu_P$ is the phytoplankton mortality rate.

\textbf{$\mathrm{De}_Z$, the zooplankton death rate}\label{sec:De_Z}

A proportion of the zooplankton in the system will die in any given time
interval. The dead zooplankton become detritus:
\begin{equation}
  \mathrm{De}_Z=\mu_Z Z^2.
\end{equation}

\textbf{$\mathrm{De}_D$, the detritus remineralisation rate}\label{sec:De_D}

As the detritus falls through the water column, it gradually converts to
nutrient:
\begin{equation}
  \mathrm{De}_D=\mu_D D.
\end{equation}

\subsection{Six-component model}
\label{sec:bio-6-component}

The six-component model is based on the model of \citet{popova2006}
which is designed to be applicably globally. 
Figure \ref{fig:biofluxes6} shows the interactions between the six
biological tracers. Nutrients, either ammonium or nitrate, (plus sunlight) are converted into
phytoplankton. Zooplankton feed on phytoplankton and detritus but in doing
so produce some nutrients and detritus so grazing also results in fluxes
from phytoplankton and detritus to nutrient and detritus. Both phytoplankton
and zooplankton die producing detritus and detritus is gradually converted to
nutrient by a re-mineralisation process. In addition, chlorophyll is present
as a subset of the phytoplankton.

\begin{figure}[ht]
  \centering
  \pdffig[width=0.7\textwidth]{embedded_models_images/bio_model_6}
  \caption{Six-component biology model.}
  \label{fig:biofluxes6}
\end{figure}


The source terms for phytoplankton ($P$), Chlorophyll-a ($\mathrm{Chl}$), zooplankton ($Z$), nitrate ($N$),
ammonium ($A$), and detritus ($D$) respectively are given by the following expressions:
\begin{gather}\label{eq:bio6_sources}
  S_P=PJ(\mathrm{Q}_N+\mathrm{Q}_A) - \mathrm{G}_P - \mathrm{De}_P,\\
  S_{\mathrm{Chl}}=(\mathrm{R}_P*J*(\mathrm{Q}_N+\mathrm{Q}_A)*P + 
      (-\mathrm{G}_P-\mathrm{De}_P))*\theta/\zeta),\\
  S_Z=\delta*(\beta_P*\mathrm{G}_P+\beta_D*\mathrm{G}_D) - \mathrm{De}_Z,\\
  S_N=-J*P*\mathrm{Q}_N+\mathrm{De}_A,\\
  S_A=-J*P*\mathrm{Q}_A + \mathrm{De}_D + (1 - \delta)*(\beta_P*\mathrm{G}_P + \beta_D*\mathrm{G}_D) + (1-\gamma)*\mathrm{De}_Z-\mathrm{De}_A,\\
  S_D=-\mathrm{De}_D + \mathrm{De}_P + \gamma*\mathrm{De}_Z +(1-\beta_P)*\mathrm{G}_P - \beta_D*\mathrm{G}_D
\end{gather}
The terms in these equations are given in table 

\ref{tab:bioparameters6}. The variable terms are explained in more detail below.
\subsubsection{Biological source terms}

\begin{table}[ht]
  \centering
  \begin{tabular}{lp{8cm}p{5cm}l}\hline
    \textbf{Symbol} & \textbf{Meaning} & \textbf{Typical value} & \textbf{Equation}\\\hline
    $\alpha$ & initial slope of $P-I$ curve,\unit[(W m${-2}$)${-1}$ day$^{-1}$] && \eqref{eq:bio6_alpha} \\
    $\alpha_c$ & Chl-a specific initial slope of $P-I$ curve & 2 \unit[(gCgChl$^{-1}$Wm$^{-2}$)${-1}$day$^{-1}$] & \\
    $\beta_P, \beta_D$ & assimilation coefficients of zooplankton & 0.75 & \\
    $\mathrm{De}_D$ & rate of breakdown of detritus to ammonium && \eqref{eq:bio6_dd}\\
    $\mathrm{De}_P$ & rate of phytoplankton natural mortality && \eqref{eq:bio6_dp}\\
    $\mathrm{De}_Z$ & rate of zooplankton natural mortality && \eqref{eq:bio6_dz} \\
    $\mathrm{De}_A$ & ammonium nitrification rate && \eqref{eq:bio6_da} \\    
    $\delta$ & excretion parameter & 0.7 & \\
    $\epsilon$ & grazing parameter relating capture of prey items to prey density & 0.4 & \\
    $\mathrm{G}_P$ & rate of zooplankton grazing on phytoplankton && \eqref{eq:bio6_gp}\\
    $\mathrm{G}_D$ & rate of zooplankton grazing on detritus && \eqref{eq:bio6_gd}\\
    $g$ & zooplankton maximum growth rate & 1.3 \unit[day$^{-1}$] & \\
    $\gamma$ & fraction of zooplankton mortality going to detritus & 0.5 & \\
    $I_0$ & photosynthetically active radiation (PAR) immediately below surface of water. Assumed to be 0.43 of the surface radiation &  \\
    $J$ & light-limited phytoplankton growth rate, \unit[day$^{-1}$] && \eqref{eq:bio6_j}\\
    $k_A$ & half-saturation constant for ammonium uptake & 0.5 \unit[mmol m$^{-3}$]  & \\
    $k_N$ & half-saturation constant for nitrate uptake & 0.5 \unit[mmol m$^{-3}$]  & \\
    $k_P$ & half-saturation constant for phytoplankton mortality & 1 \unit[mmol m$^{-3}$]  & \\
    $k_Z$ & half-saturation constant for zooplankton mortality & 3 \unit[mmol m$^{-3}$]  & \\
    $k_w$ & light attenuation due to water & 0.04 \unit[m$^{-1}$] & \\ 
    $k_c$ & light attenuation due to phytoplankton & 0.03 \unit[m$^2$ mmol$^{-1}$] & \\ 
    $\lambda_{\mathrm{bio}}$ & rate of the phytoplankton and zooplankton transfer into detritus & 0.05 \unit[day$^{-1}$]& \\
    $\lambda_A$ & nitrification rate  & 0.03 \unit[day$^{-1}$]& \\
    $\mu_P$ & phytoplankton mortality rate & 0.05 \unit[day$^{-1}$] & \\
    $\mu_Z$ & zooplankton mortality rate & 0.2 \unit[day$^{-1}$] & \\
    $\mu_D$ & detritus reference mineralisaiton rate & 0.05 \unit[day$^{-1}$] & \\
    $\Psi$ & strength of ammonium inhibition of nitrate uptake & 2.9 \unit[(mmol m$^{-3}$)$^{-1}$] & \\
    $p_P$ & relative grazing preference for phytoplankton & 0.75 & \\
    $p_D$ & relative grazing preference for detritus & 0.25 & \\
    $\mathrm{Q}_N$ & non-dimensional nitrate limiting factor && \eqref{eq:bio6_qn}  \\
    $\mathrm{Q}_A$ & non-dimensional ammonium limiting factor && \eqref{eq:bio6_qa}  \\
    $\mathrm{R}_P$ & Chl growth scaling factor && \eqref{eq:bio6_chlgrowth} \\
    $v$ & Maximum phytoplankton growth rate & 1 \unit[day$^{-1}$] & \\
    $w_g$ & detritus sinking velocity & 10 \unit[m day$^{-1}$]& \\
    $z$ & depth && \\
    $\theta$ & Chl to carbon ratio, \unit[mg Chl mgC$^{-1}$] && \\
    $\theta_m$ & maximum Chl to carbon ratio & 0.05 \unit[mg Chl mgC$^{-1}$] &\\
    $\zeta$ & conversion factor from \unit[gC] to \unit[mmolN] based on C:N ratio of 6.5 & 0.0128 \unit[mmolN ngC$^{-1}$] & \\
    \hline
  \end{tabular}
  \caption{Meanings of symbols in the six-component biology model. Typical values are
    provided for externally set parameters.}
  \label{tab:bioparameters6}
\end{table}

Unlike the model of \citet{popova2006} we use a continuous model, with no
change of equations (bar one exception) above or below the photic zone. For
our purposes, the photic zone is defined as 100m water depth. First we calculate
$\theta$:
\begin{equation}
    \theta = \frac{\mathrm{Chl}}{\mathrm{P}\zeta}
\label{eq:bio6_theta}
\end{equation}

However, at low light levels, $\mathrm{Chl}$ might be zero, therefore
we take the limit that $\theta \rightarrow \zeta$ at low levels ($1e^{-7}$) of
chlorphyll-a or photoplankton.

We then calculate $\alpha$:
\begin{equation}
    alpha = \alpha_c \theta
\label{eq:bio6_alpha}
\end{equation}

Using the PAR available at each vertex of the mesh, we now calculate the
light-limited phytoplankton growth rate, $J$:
\begin{equation}
    J = \frac{v\alpha I_n}{\sqrt{v^2 + \alpha^2 + I_n^2}}
\label{eq:bio6_j}
\end{equation}

The limiting factors on nitrate and ammonium are now calculated:
\begin{equation}
    \mathrm{Q}_N = \frac{N\exp^{-\Psi A}}{K_N+N}\label{eq:bio6_qn}, 
\end{equation}
\begin{equation}
    \mathrm{Q}_A = \frac{A}{K_A+A}\label{eq:bio6_qa}
\end{equation}

From these the diagnostic field, primary production ($\mathrm{X_P}$), can be calculated:
\begin{equation}
    \mathrm{X_P}=J\left(Q_N + Q_A\right)P
\label{eq:bio6_primprod}
\end{equation}

The chlorophyll growth scaling factor is given by:
\begin{equation}
    \mathrm{R}_P = Q_N Q_A \left(\frac{\theta_m}{\theta}\right) \left(\frac{v}{\sqrt{v^2 + \alpha^2 + I_n^2}}\right)
\label{eq:bio6_chlgrowth}
\end{equation}

The zooplankton grazing terms are now calculated:
\begin{equation}
    \mathrm{G}_P = \frac{gp_PP^2Z}{\epsilon + \left(p_PP^2 + p_DD^2\right)}\label{eq:bio6_gp},
\end{equation}
\begin{equation}
    \mathrm{G}_D = \frac{gp_DD^2*Z}{\epsilon+\left(p_PP^2 + p_DD^2\right)}\label{eq:bio6_gd}
\end{equation}

Finally, the four death rates and re-mineralisation rates are calculated:
\begin{equation}
    \mathrm{De}_P = \frac{\mu_pP^2}{P+k_p} + \lambda_{\mathrm{bio}}*P\label{eq:bio6_dp},
\end{equation}
\begin{equation}
    \mathrm{De}_Z = \frac{\mu_zZ^3}{Z+k_z} + \lambda_{\mathrm{bio}}*Z\label{eq:bio6_dz},
\end{equation}
\begin{equation}
    \mathrm{De}_D = \mu_DD+\lambda_{\mathrm{bio}}*P + \lambda_{\mathrm{bio}}*Z\label{eq:bio6_dd},
\end{equation}
\begin{equation}
    \mathrm{De}_A = \lambda_A A \; \mbox{where $z < 100$}\label{eq:bio6_da}
\end{equation}


\subsection{Photosynthetically active radiation (PAR)}\label{sec:I}

Phytoplankton depends on the levels of light in the water column at the
frequencies useful for photosynthesis. This is referred to as
photosynthetically active radiation. Sunlight falls on the water surface and
is absorbed by both water and phytoplankton. This is represented by the
following equation:
\begin{equation}
  \ppx[z]{I}=(A_{\mathrm{water}}+A_PP)I,
\end{equation}
where $A_{\mathrm{water}}$ and $A_P$ are the absorption rates of
photosynthetically active radiation by water and phytoplankton respectively.

This equation has the form of a one-dimensional advection equation and
therefore requires a single boundary condition: the amount of light incident
on the water surface. This is a Dirichlet condition on $I$.

\subsection{Detritus falling velocity}\label{sec:detritus}

Phytoplankton, zooplankton and nutrient are assumed to be neutrally buoyant
and therefore move only under the advecting velocity of the water or by
diffusive mixing. Detritus, on the other hand, is assumed to be denser than
water so it will slowly sink through the water-column. This is modelled by
modifying the advecting velocity in the advection-diffusion equation for
detritus by subtracting a sinking velocity $u_{\mathrm{sink}}$ from the
vertical component of the advecting velocity.

\section{Sediments}

\fluidity\ is capable of simulating an unlimited number of sediment concentration classes.
Each class has a separate grain size, density and settling velocity. The sediment behaves
as any other tracer field, except it is subject to a settling velocity:

\begin{equation}\label{eq:sediment_conc}
\ppt{S_m} + \nabla\cdot(\bmu S_m) = \nabla\cdot(\kaptens\nabla S_m) - \sigma S_m
\end{equation}

The advection in the vertical direction is then modified with a downwards sinking 
velocity $u_{\mathrm{sink}}$. 

\subsection{Deposition and erosion}

A surface can be defined, the sea-bed, which is a sink for sediment. Once sediment
fluxes through this surface it is removed from the system and stored in a separate 
field: the SedimentFlux field. Each sediment class has an equivalent Flux field.

Erosion occurs when the bed-shear stress is greater than the critical shear stress. Each
sediment class has a separate shear stress, which can be input or calculated depending
on the options chosen. Erosion flux, $E_m$ is implemented as a Neumann boundary condition
on the deposition/erosion surface.

\begin{equation}\label{eq:sediment_erosion_rate}
E_m = E_{0m}\left(1-\varphi\right)\frac{\tau_{sf} - \tau_{cm}}{\tau_{cm}}
\end{equation}

\noindent
where $E_{0m}$ is the bed erodibility constant (kgm$^{-1}$s${-1}$) for sediment class $m$,
$\tau_{sf}$ is the bed-shear stress, $\varphi$ is the bed porosity (typically 0.3)
and $\tau_{cm}$ is the critical shear stress
for sediment class $m$. The critical shear stress can be input by the user or 
automatically calculated using:

\begin{equation}\label{eq:critical_shear_stress}
\tau_{cm} = 0.041\left(s-1\right)\rho gD
\end{equation}

\noindent
where s is the relative density of the sediment, i.e. $\frac{\rho_{S_{m}}}{\rho}$ and $D$ is the sediment
diameter (mm). The SedimentFlux field effectively mixes the deposited sediment, so order of deposition
is not preserved.

