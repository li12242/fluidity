\documentclass{article}
%\documentclass{svjour3}
\usepackage{amsthm}
\usepackage[sumlimits]{amsmath}
\usepackage{amsfonts}
\usepackage{color}
%\usepackage{lineno}
\usepackage{graphicx}
\DeclareMathOperator{\ad}{ad}
\DeclareMathOperator{\Vol}{Vol}
\DeclareMathOperator{\diff}{d}
\DeclareMathOperator{\Emb}{Emb}
\DeclareMathOperator{\Diff}{Diff}
\DeclareMathOperator{\Id}{Id}
\DeclareMathOperator{\Tr}{Tr}

\newtheorem{theorem}{Theorem}
\newtheorem{lemma}[theorem]{Lemma}
\newtheorem{corollary}[theorem]{Corollary}
\newtheorem{definition}[theorem]{Definition}
\newtheorem{example}[theorem]{Example}
\newtheorem{proposition}[theorem]{Proposition}
\newtheorem{remark}[theorem]{Remark}
\def\MM#1{\boldsymbol{#1}}
\newcommand{\pp}[2]{\frac{\partial #1}{\partial #2}} 
\newcommand{\dede}[2]{\frac{\delta #1}{\delta #2}}
\newcommand{\dd}[2]{\frac{\diff#1}{\diff #2}} 
\newcommand{\p}{\MM{P}}
\newcommand{\eqnref}[1]{(\ref{#1})}
\newcommand{\q}{\MM{Q}}
\newcommand{\U}{\MM{u}}
\newcommand{\sothree}{\mathfrak{so}(3)}
\newcommand{\logapp}{\overline{\log}}
\newcommand{\expapp}{\overline{\exp}}
\newcommand{\bfi}[1]{{\bfseries\itshape #1}}
\newcommand{\comment}[1]{\vspace{1 mm}\par
\marginpar{\large\underline{}}\noindent
\framebox{\begin{minipage}[c]{0.95 \textwidth}
\color{blue}\bfi #1 \end{minipage}}\vspace{2 mm}\par}
\bibliographystyle{alpha}
\begin{document}
\title{Finite element formulation of simple adjoint problem}
\author{cjc}
\maketitle

(Note from pef: this note has errors of algebra in it that I have not bothered to correct.)

The strong form of the forward problem is
\[
u_{xx} = 0, \, u(0)=a, \, u(1) = b.
\]
For the functional $J=u_x|_b$ the Lagrangian with Lagrange multipliers
for boundary conditions is
\[
L[u,\lambda,\mu] = u_x|_1 + \langle \lambda_x,u_x\rangle - 
[\lambda u_x]_0^1 + \mu_0(u|_0-a) + \mu_1(u|_1-b).
\]
Variations then yield the following equations:
\begin{eqnarray}
\label{u eqn}
\langle \delta\lambda_x,u_x\rangle - [\delta\lambda u_x]_0^1 & = & 0 \\
\label{u0}
u_0 & = & a\\
\label{u1}
u_1 & = & b\\
\label{lambda}
\left(\delta u\right)_x|_1 + \langle \lambda_x,(\delta u)_x\rangle -
[\lambda (\delta u)_x]_0^1 + \mu_0(\delta u)|_0 + \mu_1(\delta u)|_1
\end{eqnarray}.
Equations (\ref{u eqn}-\ref{u1}) have the unique solution
\[
u = a + x(a-b).
\]
Equation \eqref{lambda} has the unique solution
\[
\lambda = x, \, \mu_0=\lambda_x|_0, \, \mu_1=\lambda_x|_1.
\]
Now discretise $u$ and $\lambda$ using P1 finite elements on an
equispaced grid, width $h$. The Lagrangian becomes
\[
L = \MM{u}^TA\MM{\lambda} + \MM{u}^TB\left(\MM{e}_N-\MM{\lambda}\right)
 + \MM{u}^TL\MM{\lambda} + \mu_0(u_0-a) + \mu_1(u_N-b),
\]
where $\MM{u}=(u_0,\ldots,u_N)$ and
$\MM{\lambda}=(\lambda_0,\ldots,\lambda_N)$ are vectors of basis
coefficients for $u$ and $\lambda$, $\MM{e}_N$ is the $N$-th basis vector,
and
\[
A_{ij} = \phi_i'(0)\phi_j(0), \quad
B_{ij} = \phi_i'(1)\phi_j(1), \quad
L_{ij} = \langle \phi_i,\phi_j\rangle.
\]
We can calculate these matrices:
\[
A = \frac{1}{h}
\begin{pmatrix}
-1 & 0 & \ldots & 0 \\
1 & 0 & \ldots & 0 \\
0 & 0 & \ldots & 0 \\
\vdots & \vdots & \ddots & \vdots \\
0 & \ldots & \ldots & 0 \\
\end{pmatrix}
,
\quad
B= \frac{1}{h}
\begin{pmatrix}
0 & \ldots & \ldots & 0 \\
\vdots & \ddots & \vdots & \vdots \\
0 & \ldots & 0 & 0 \\
0 & \ldots & 0 & -1\\
0 & \ldots & 0 & 1\\
\end{pmatrix},
\]
and
\[
L=
\frac{1}{h}
\begin{pmatrix}
1 & -1 &        &      & & \\
-1 & 2 & -1  &      & & \\
 & \ddots & \ddots & \ddots & & \\
 & & \ddots & \ddots & \ddots & \\
 & &        & -1 & 2 & -1 \\
 & &        &        & -1     & 1 \\
\end{pmatrix}
\]
The equations are
\begin{eqnarray}
\label{fe u eqn}
L\MM{u} & = & (B^T-A^T)\MM{u} \\
\label{fe u0}
u_0 & = & a\\
\label{fe u1}
u_1 & = & b\\
\label{fe lambda}
L^T\MM{\lambda} & = & -A\MM{\lambda} + B(\MM{\lambda}-\MM{e}_N)
+ \mu_0\MM{e}_0 - \mu_1\MM{e}_N.
\end{eqnarray}.
Solution is as follows:
\begin{enumerate}
\item We immediately know the boundary conditions from (\ref{fe
  u0}-\ref{fe u1}).
\item The $A^T$ and $B^T$ remove the $0$ and $N$ rows from $L$ and
  turn those rows into the trivial equation $0=0$.
\item We solve for $(u_1,\ldots,u_{N-1})$ using the remaining rows
  with known $u_0$ and $u_N$ appearing on the RHS.
\item The $A$ and $B$ matrices in the $\lambda$ equation remove the
  $0$ and $N$ columns, setting $\lambda_0=0$ and $\lambda_1=1$.
\item The multipliers $\mu_0$ and $\mu_1$ are diagnose from the 0 and
  $N$ rows.
\end{enumerate}

\end{document}

